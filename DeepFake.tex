\documentclass[12pt]{article}

    \usepackage[breakable]{tcolorbox}
    \usepackage{parskip} % Stop auto-indenting (to mimic markdown behaviour)
    

    % Basic figure setup, for now with no caption control since it's done
    % automatically by Pandoc (which extracts ![](path) syntax from Markdown).
    \usepackage{graphicx}
    % Keep aspect ratio if custom image width or height is specified
    \setkeys{Gin}{keepaspectratio}
    % Maintain compatibility with old templates. Remove in nbconvert 6.0
    \let\Oldincludegraphics\includegraphics
    % Ensure that by default, figures have no caption (until we provide a
    % proper Figure object with a Caption API and a way to capture that
    % in the conversion process - todo).
    \usepackage{caption}
    \DeclareCaptionFormat{nocaption}{}
    \captionsetup{format=nocaption,aboveskip=0pt,belowskip=0pt}

    \usepackage{float}
    \floatplacement{figure}{H} % forces figures to be placed at the correct location
    \usepackage{xcolor} % Allow colors to be defined
    \usepackage{enumerate} % Needed for markdown enumerations to work
    \usepackage{geometry} % Used to adjust the document margins
    \usepackage{amsmath} % Equations
    \usepackage{amssymb} % Equations
    \usepackage{textcomp} % defines textquotesingle
    % Hack from http://tex.stackexchange.com/a/47451/13684:
    \AtBeginDocument{%
        \def\PYZsq{\textquotesingle}% Upright quotes in Pygmentized code
    }
    \usepackage{upquote} % Upright quotes for verbatim code
    \usepackage{eurosym} % defines \euro

    \usepackage{iftex}
    \ifPDFTeX
        \usepackage[T1]{fontenc}
        \IfFileExists{alphabeta.sty}{
              \usepackage{alphabeta}
          }{
              \usepackage[mathletters]{ucs}
              \usepackage[utf8x]{inputenc}
          }
    \else
        \usepackage{fontspec}
        \usepackage{unicode-math}
    \fi

    \usepackage{fancyvrb} % verbatim replacement that allows latex
    \usepackage{grffile} % extends the file name processing of package graphics
                         % to support a larger range
    \makeatletter % fix for old versions of grffile with XeLaTeX
    \@ifpackagelater{grffile}{2019/11/01}
    {
      % Do nothing on new versions
    }
    {
      \def\Gread@@xetex#1{%
        \IfFileExists{"\Gin@base".bb}%
        {\Gread@eps{\Gin@base.bb}}%
        {\Gread@@xetex@aux#1}%
      }
    }
    \makeatother
    \usepackage[Export]{adjustbox} % Used to constrain images to a maximum size
    \adjustboxset{max size={0.9\linewidth}{0.9\paperheight}}

    % The hyperref package gives us a pdf with properly built
    % internal navigation ('pdf bookmarks' for the table of contents,
    % internal cross-reference links, web links for URLs, etc.)
    \usepackage{hyperref}
    % The default LaTeX title has an obnoxious amount of whitespace. By default,
    % titling removes some of it. It also provides customization options.
    \usepackage{titling}
    \usepackage{longtable} % longtable support required by pandoc >1.10
    \usepackage{booktabs}  % table support for pandoc > 1.12.2
    \usepackage{array}     % table support for pandoc >= 2.11.3
    \usepackage{calc}      % table minipage width calculation for pandoc >= 2.11.1
    \usepackage[inline]{enumitem} % IRkernel/repr support (it uses the enumerate* environment)
    \usepackage[normalem]{ulem} % ulem is needed to support strikethroughs (\sout)
                                % normalem makes italics be italics, not underlines
    \usepackage{soul}      % strikethrough (\st) support for pandoc >= 3.0.0
    \usepackage{mathrsfs}
    

    
    % Colors for the hyperref package
    \definecolor{urlcolor}{rgb}{0,.145,.698}
    \definecolor{linkcolor}{rgb}{.71,0.21,0.01}
    \definecolor{citecolor}{rgb}{.12,.54,.11}

    % ANSI colors
    \definecolor{ansi-black}{HTML}{3E424D}
    \definecolor{ansi-black-intense}{HTML}{282C36}
    \definecolor{ansi-red}{HTML}{E75C58}
    \definecolor{ansi-red-intense}{HTML}{B22B31}
    \definecolor{ansi-green}{HTML}{00A250}
    \definecolor{ansi-green-intense}{HTML}{007427}
    \definecolor{ansi-yellow}{HTML}{DDB62B}
    \definecolor{ansi-yellow-intense}{HTML}{B27D12}
    \definecolor{ansi-blue}{HTML}{208FFB}
    \definecolor{ansi-blue-intense}{HTML}{0065CA}
    \definecolor{ansi-magenta}{HTML}{D160C4}
    \definecolor{ansi-magenta-intense}{HTML}{A03196}
    \definecolor{ansi-cyan}{HTML}{60C6C8}
    \definecolor{ansi-cyan-intense}{HTML}{258F8F}
    \definecolor{ansi-white}{HTML}{C5C1B4}
    \definecolor{ansi-white-intense}{HTML}{A1A6B2}
    \definecolor{ansi-default-inverse-fg}{HTML}{FFFFFF}
    \definecolor{ansi-default-inverse-bg}{HTML}{000000}

    % common color for the border for error outputs.
    \definecolor{outerrorbackground}{HTML}{FFDFDF}

    % commands and environments needed by pandoc snippets
    % extracted from the output of `pandoc -s`
    \providecommand{\tightlist}{%
      \setlength{\itemsep}{0pt}\setlength{\parskip}{0pt}}
    \DefineVerbatimEnvironment{Highlighting}{Verbatim}{commandchars=\\\{\}}
    % Add ',fontsize=\small' for more characters per line
    \newenvironment{Shaded}{}{}
    \newcommand{\KeywordTok}[1]{\textcolor[rgb]{0.00,0.44,0.13}{\textbf{{#1}}}}
    \newcommand{\DataTypeTok}[1]{\textcolor[rgb]{0.56,0.13,0.00}{{#1}}}
    \newcommand{\DecValTok}[1]{\textcolor[rgb]{0.25,0.63,0.44}{{#1}}}
    \newcommand{\BaseNTok}[1]{\textcolor[rgb]{0.25,0.63,0.44}{{#1}}}
    \newcommand{\FloatTok}[1]{\textcolor[rgb]{0.25,0.63,0.44}{{#1}}}
    \newcommand{\CharTok}[1]{\textcolor[rgb]{0.25,0.44,0.63}{{#1}}}
    \newcommand{\StringTok}[1]{\textcolor[rgb]{0.25,0.44,0.63}{{#1}}}
    \newcommand{\CommentTok}[1]{\textcolor[rgb]{0.38,0.63,0.69}{\textit{{#1}}}}
    \newcommand{\OtherTok}[1]{\textcolor[rgb]{0.00,0.44,0.13}{{#1}}}
    \newcommand{\AlertTok}[1]{\textcolor[rgb]{1.00,0.00,0.00}{\textbf{{#1}}}}
    \newcommand{\FunctionTok}[1]{\textcolor[rgb]{0.02,0.16,0.49}{{#1}}}
    \newcommand{\RegionMarkerTok}[1]{{#1}}
    \newcommand{\ErrorTok}[1]{\textcolor[rgb]{1.00,0.00,0.00}{\textbf{{#1}}}}
    \newcommand{\NormalTok}[1]{{#1}}

    % Additional commands for more recent versions of Pandoc
    \newcommand{\ConstantTok}[1]{\textcolor[rgb]{0.53,0.00,0.00}{{#1}}}
    \newcommand{\SpecialCharTok}[1]{\textcolor[rgb]{0.25,0.44,0.63}{{#1}}}
    \newcommand{\VerbatimStringTok}[1]{\textcolor[rgb]{0.25,0.44,0.63}{{#1}}}
    \newcommand{\SpecialStringTok}[1]{\textcolor[rgb]{0.73,0.40,0.53}{{#1}}}
    \newcommand{\ImportTok}[1]{{#1}}
    \newcommand{\DocumentationTok}[1]{\textcolor[rgb]{0.73,0.13,0.13}{\textit{{#1}}}}
    \newcommand{\AnnotationTok}[1]{\textcolor[rgb]{0.38,0.63,0.69}{\textbf{\textit{{#1}}}}}
    \newcommand{\CommentVarTok}[1]{\textcolor[rgb]{0.38,0.63,0.69}{\textbf{\textit{{#1}}}}}
    \newcommand{\VariableTok}[1]{\textcolor[rgb]{0.10,0.09,0.49}{{#1}}}
    \newcommand{\ControlFlowTok}[1]{\textcolor[rgb]{0.00,0.44,0.13}{\textbf{{#1}}}}
    \newcommand{\OperatorTok}[1]{\textcolor[rgb]{0.40,0.40,0.40}{{#1}}}
    \newcommand{\BuiltInTok}[1]{{#1}}
    \newcommand{\ExtensionTok}[1]{{#1}}
    \newcommand{\PreprocessorTok}[1]{\textcolor[rgb]{0.74,0.48,0.00}{{#1}}}
    \newcommand{\AttributeTok}[1]{\textcolor[rgb]{0.49,0.56,0.16}{{#1}}}
    \newcommand{\InformationTok}[1]{\textcolor[rgb]{0.38,0.63,0.69}{\textbf{\textit{{#1}}}}}
    \newcommand{\WarningTok}[1]{\textcolor[rgb]{0.38,0.63,0.69}{\textbf{\textit{{#1}}}}}


    % Define a nice break command that doesn't care if a line doesn't already
    % exist.
    \def\br{\hspace*{\fill} \\* }
    % Math Jax compatibility definitions
    \def\gt{>}
    \def\lt{<}
    \let\Oldtex\TeX
    \let\Oldlatex\LaTeX
    \renewcommand{\TeX}{\textrm{\Oldtex}}
    \renewcommand{\LaTeX}{\textrm{\Oldlatex}}
    % Document parameters
    % Document title
    \title{DeepFake}
    
    
    
    
    
    
    
% Pygments definitions
\makeatletter
\def\PY@reset{\let\PY@it=\relax \let\PY@bf=\relax%
    \let\PY@ul=\relax \let\PY@tc=\relax%
    \let\PY@bc=\relax \let\PY@ff=\relax}
\def\PY@tok#1{\csname PY@tok@#1\endcsname}
\def\PY@toks#1+{\ifx\relax#1\empty\else%
    \PY@tok{#1}\expandafter\PY@toks\fi}
\def\PY@do#1{\PY@bc{\PY@tc{\PY@ul{%
    \PY@it{\PY@bf{\PY@ff{#1}}}}}}}
\def\PY#1#2{\PY@reset\PY@toks#1+\relax+\PY@do{#2}}

\@namedef{PY@tok@w}{\def\PY@tc##1{\textcolor[rgb]{0.73,0.73,0.73}{##1}}}
\@namedef{PY@tok@c}{\let\PY@it=\textit\def\PY@tc##1{\textcolor[rgb]{0.24,0.48,0.48}{##1}}}
\@namedef{PY@tok@cp}{\def\PY@tc##1{\textcolor[rgb]{0.61,0.40,0.00}{##1}}}
\@namedef{PY@tok@k}{\let\PY@bf=\textbf\def\PY@tc##1{\textcolor[rgb]{0.00,0.50,0.00}{##1}}}
\@namedef{PY@tok@kp}{\def\PY@tc##1{\textcolor[rgb]{0.00,0.50,0.00}{##1}}}
\@namedef{PY@tok@kt}{\def\PY@tc##1{\textcolor[rgb]{0.69,0.00,0.25}{##1}}}
\@namedef{PY@tok@o}{\def\PY@tc##1{\textcolor[rgb]{0.40,0.40,0.40}{##1}}}
\@namedef{PY@tok@ow}{\let\PY@bf=\textbf\def\PY@tc##1{\textcolor[rgb]{0.67,0.13,1.00}{##1}}}
\@namedef{PY@tok@nb}{\def\PY@tc##1{\textcolor[rgb]{0.00,0.50,0.00}{##1}}}
\@namedef{PY@tok@nf}{\def\PY@tc##1{\textcolor[rgb]{0.00,0.00,1.00}{##1}}}
\@namedef{PY@tok@nc}{\let\PY@bf=\textbf\def\PY@tc##1{\textcolor[rgb]{0.00,0.00,1.00}{##1}}}
\@namedef{PY@tok@nn}{\let\PY@bf=\textbf\def\PY@tc##1{\textcolor[rgb]{0.00,0.00,1.00}{##1}}}
\@namedef{PY@tok@ne}{\let\PY@bf=\textbf\def\PY@tc##1{\textcolor[rgb]{0.80,0.25,0.22}{##1}}}
\@namedef{PY@tok@nv}{\def\PY@tc##1{\textcolor[rgb]{0.10,0.09,0.49}{##1}}}
\@namedef{PY@tok@no}{\def\PY@tc##1{\textcolor[rgb]{0.53,0.00,0.00}{##1}}}
\@namedef{PY@tok@nl}{\def\PY@tc##1{\textcolor[rgb]{0.46,0.46,0.00}{##1}}}
\@namedef{PY@tok@ni}{\let\PY@bf=\textbf\def\PY@tc##1{\textcolor[rgb]{0.44,0.44,0.44}{##1}}}
\@namedef{PY@tok@na}{\def\PY@tc##1{\textcolor[rgb]{0.41,0.47,0.13}{##1}}}
\@namedef{PY@tok@nt}{\let\PY@bf=\textbf\def\PY@tc##1{\textcolor[rgb]{0.00,0.50,0.00}{##1}}}
\@namedef{PY@tok@nd}{\def\PY@tc##1{\textcolor[rgb]{0.67,0.13,1.00}{##1}}}
\@namedef{PY@tok@s}{\def\PY@tc##1{\textcolor[rgb]{0.73,0.13,0.13}{##1}}}
\@namedef{PY@tok@sd}{\let\PY@it=\textit\def\PY@tc##1{\textcolor[rgb]{0.73,0.13,0.13}{##1}}}
\@namedef{PY@tok@si}{\let\PY@bf=\textbf\def\PY@tc##1{\textcolor[rgb]{0.64,0.35,0.47}{##1}}}
\@namedef{PY@tok@se}{\let\PY@bf=\textbf\def\PY@tc##1{\textcolor[rgb]{0.67,0.36,0.12}{##1}}}
\@namedef{PY@tok@sr}{\def\PY@tc##1{\textcolor[rgb]{0.64,0.35,0.47}{##1}}}
\@namedef{PY@tok@ss}{\def\PY@tc##1{\textcolor[rgb]{0.10,0.09,0.49}{##1}}}
\@namedef{PY@tok@sx}{\def\PY@tc##1{\textcolor[rgb]{0.00,0.50,0.00}{##1}}}
\@namedef{PY@tok@m}{\def\PY@tc##1{\textcolor[rgb]{0.40,0.40,0.40}{##1}}}
\@namedef{PY@tok@gh}{\let\PY@bf=\textbf\def\PY@tc##1{\textcolor[rgb]{0.00,0.00,0.50}{##1}}}
\@namedef{PY@tok@gu}{\let\PY@bf=\textbf\def\PY@tc##1{\textcolor[rgb]{0.50,0.00,0.50}{##1}}}
\@namedef{PY@tok@gd}{\def\PY@tc##1{\textcolor[rgb]{0.63,0.00,0.00}{##1}}}
\@namedef{PY@tok@gi}{\def\PY@tc##1{\textcolor[rgb]{0.00,0.52,0.00}{##1}}}
\@namedef{PY@tok@gr}{\def\PY@tc##1{\textcolor[rgb]{0.89,0.00,0.00}{##1}}}
\@namedef{PY@tok@ge}{\let\PY@it=\textit}
\@namedef{PY@tok@gs}{\let\PY@bf=\textbf}
\@namedef{PY@tok@ges}{\let\PY@bf=\textbf\let\PY@it=\textit}
\@namedef{PY@tok@gp}{\let\PY@bf=\textbf\def\PY@tc##1{\textcolor[rgb]{0.00,0.00,0.50}{##1}}}
\@namedef{PY@tok@go}{\def\PY@tc##1{\textcolor[rgb]{0.44,0.44,0.44}{##1}}}
\@namedef{PY@tok@gt}{\def\PY@tc##1{\textcolor[rgb]{0.00,0.27,0.87}{##1}}}
\@namedef{PY@tok@err}{\def\PY@bc##1{{\setlength{\fboxsep}{\string -\fboxrule}\fcolorbox[rgb]{1.00,0.00,0.00}{1,1,1}{\strut ##1}}}}
\@namedef{PY@tok@kc}{\let\PY@bf=\textbf\def\PY@tc##1{\textcolor[rgb]{0.00,0.50,0.00}{##1}}}
\@namedef{PY@tok@kd}{\let\PY@bf=\textbf\def\PY@tc##1{\textcolor[rgb]{0.00,0.50,0.00}{##1}}}
\@namedef{PY@tok@kn}{\let\PY@bf=\textbf\def\PY@tc##1{\textcolor[rgb]{0.00,0.50,0.00}{##1}}}
\@namedef{PY@tok@kr}{\let\PY@bf=\textbf\def\PY@tc##1{\textcolor[rgb]{0.00,0.50,0.00}{##1}}}
\@namedef{PY@tok@bp}{\def\PY@tc##1{\textcolor[rgb]{0.00,0.50,0.00}{##1}}}
\@namedef{PY@tok@fm}{\def\PY@tc##1{\textcolor[rgb]{0.00,0.00,1.00}{##1}}}
\@namedef{PY@tok@vc}{\def\PY@tc##1{\textcolor[rgb]{0.10,0.09,0.49}{##1}}}
\@namedef{PY@tok@vg}{\def\PY@tc##1{\textcolor[rgb]{0.10,0.09,0.49}{##1}}}
\@namedef{PY@tok@vi}{\def\PY@tc##1{\textcolor[rgb]{0.10,0.09,0.49}{##1}}}
\@namedef{PY@tok@vm}{\def\PY@tc##1{\textcolor[rgb]{0.10,0.09,0.49}{##1}}}
\@namedef{PY@tok@sa}{\def\PY@tc##1{\textcolor[rgb]{0.73,0.13,0.13}{##1}}}
\@namedef{PY@tok@sb}{\def\PY@tc##1{\textcolor[rgb]{0.73,0.13,0.13}{##1}}}
\@namedef{PY@tok@sc}{\def\PY@tc##1{\textcolor[rgb]{0.73,0.13,0.13}{##1}}}
\@namedef{PY@tok@dl}{\def\PY@tc##1{\textcolor[rgb]{0.73,0.13,0.13}{##1}}}
\@namedef{PY@tok@s2}{\def\PY@tc##1{\textcolor[rgb]{0.73,0.13,0.13}{##1}}}
\@namedef{PY@tok@sh}{\def\PY@tc##1{\textcolor[rgb]{0.73,0.13,0.13}{##1}}}
\@namedef{PY@tok@s1}{\def\PY@tc##1{\textcolor[rgb]{0.73,0.13,0.13}{##1}}}
\@namedef{PY@tok@mb}{\def\PY@tc##1{\textcolor[rgb]{0.40,0.40,0.40}{##1}}}
\@namedef{PY@tok@mf}{\def\PY@tc##1{\textcolor[rgb]{0.40,0.40,0.40}{##1}}}
\@namedef{PY@tok@mh}{\def\PY@tc##1{\textcolor[rgb]{0.40,0.40,0.40}{##1}}}
\@namedef{PY@tok@mi}{\def\PY@tc##1{\textcolor[rgb]{0.40,0.40,0.40}{##1}}}
\@namedef{PY@tok@il}{\def\PY@tc##1{\textcolor[rgb]{0.40,0.40,0.40}{##1}}}
\@namedef{PY@tok@mo}{\def\PY@tc##1{\textcolor[rgb]{0.40,0.40,0.40}{##1}}}
\@namedef{PY@tok@ch}{\let\PY@it=\textit\def\PY@tc##1{\textcolor[rgb]{0.24,0.48,0.48}{##1}}}
\@namedef{PY@tok@cm}{\let\PY@it=\textit\def\PY@tc##1{\textcolor[rgb]{0.24,0.48,0.48}{##1}}}
\@namedef{PY@tok@cpf}{\let\PY@it=\textit\def\PY@tc##1{\textcolor[rgb]{0.24,0.48,0.48}{##1}}}
\@namedef{PY@tok@c1}{\let\PY@it=\textit\def\PY@tc##1{\textcolor[rgb]{0.24,0.48,0.48}{##1}}}
\@namedef{PY@tok@cs}{\let\PY@it=\textit\def\PY@tc##1{\textcolor[rgb]{0.24,0.48,0.48}{##1}}}

\def\PYZbs{\char`\\}
\def\PYZus{\char`\_}
\def\PYZob{\char`\{}
\def\PYZcb{\char`\}}
\def\PYZca{\char`\^}
\def\PYZam{\char`\&}
\def\PYZlt{\char`\<}
\def\PYZgt{\char`\>}
\def\PYZsh{\char`\#}
\def\PYZpc{\char`\%}
\def\PYZdl{\char`\$}
\def\PYZhy{\char`\-}
\def\PYZsq{\char`\'}
\def\PYZdq{\char`\"}
\def\PYZti{\char`\~}
% for compatibility with earlier versions
\def\PYZat{@}
\def\PYZlb{[}
\def\PYZrb{]}
\makeatother


    % For linebreaks inside Verbatim environment from package fancyvrb.
    \makeatletter
        \newbox\Wrappedcontinuationbox
        \newbox\Wrappedvisiblespacebox
        \newcommand*\Wrappedvisiblespace {\textcolor{red}{\textvisiblespace}}
        \newcommand*\Wrappedcontinuationsymbol {\textcolor{red}{\llap{\tiny$\m@th\hookrightarrow$}}}
        \newcommand*\Wrappedcontinuationindent {3ex }
        \newcommand*\Wrappedafterbreak {\kern\Wrappedcontinuationindent\copy\Wrappedcontinuationbox}
        % Take advantage of the already applied Pygments mark-up to insert
        % potential linebreaks for TeX processing.
        %        {, <, #, %, $, ' and ": go to next line.
        %        _, }, ^, &, >, - and ~: stay at end of broken line.
        % Use of \textquotesingle for straight quote.
        \newcommand*\Wrappedbreaksatspecials {%
            \def\PYGZus{\discretionary{\char`\_}{\Wrappedafterbreak}{\char`\_}}%
            \def\PYGZob{\discretionary{}{\Wrappedafterbreak\char`\{}{\char`\{}}%
            \def\PYGZcb{\discretionary{\char`\}}{\Wrappedafterbreak}{\char`\}}}%
            \def\PYGZca{\discretionary{\char`\^}{\Wrappedafterbreak}{\char`\^}}%
            \def\PYGZam{\discretionary{\char`\&}{\Wrappedafterbreak}{\char`\&}}%
            \def\PYGZlt{\discretionary{}{\Wrappedafterbreak\char`\<}{\char`\<}}%
            \def\PYGZgt{\discretionary{\char`\>}{\Wrappedafterbreak}{\char`\>}}%
            \def\PYGZsh{\discretionary{}{\Wrappedafterbreak\char`\#}{\char`\#}}%
            \def\PYGZpc{\discretionary{}{\Wrappedafterbreak\char`\%}{\char`\%}}%
            \def\PYGZdl{\discretionary{}{\Wrappedafterbreak\char`\$}{\char`\$}}%
            \def\PYGZhy{\discretionary{\char`\-}{\Wrappedafterbreak}{\char`\-}}%
            \def\PYGZsq{\discretionary{}{\Wrappedafterbreak\textquotesingle}{\textquotesingle}}%
            \def\PYGZdq{\discretionary{}{\Wrappedafterbreak\char`\"}{\char`\"}}%
            \def\PYGZti{\discretionary{\char`\~}{\Wrappedafterbreak}{\char`\~}}%
        }
        % Some characters . , ; ? ! / are not pygmentized.
        % This macro makes them "active" and they will insert potential linebreaks
        \newcommand*\Wrappedbreaksatpunct {%
            \lccode`\~`\.\lowercase{\def~}{\discretionary{\hbox{\char`\.}}{\Wrappedafterbreak}{\hbox{\char`\.}}}%
            \lccode`\~`\,\lowercase{\def~}{\discretionary{\hbox{\char`\,}}{\Wrappedafterbreak}{\hbox{\char`\,}}}%
            \lccode`\~`\;\lowercase{\def~}{\discretionary{\hbox{\char`\;}}{\Wrappedafterbreak}{\hbox{\char`\;}}}%
            \lccode`\~`\:\lowercase{\def~}{\discretionary{\hbox{\char`\:}}{\Wrappedafterbreak}{\hbox{\char`\:}}}%
            \lccode`\~`\?\lowercase{\def~}{\discretionary{\hbox{\char`\?}}{\Wrappedafterbreak}{\hbox{\char`\?}}}%
            \lccode`\~`\!\lowercase{\def~}{\discretionary{\hbox{\char`\!}}{\Wrappedafterbreak}{\hbox{\char`\!}}}%
            \lccode`\~`\/\lowercase{\def~}{\discretionary{\hbox{\char`\/}}{\Wrappedafterbreak}{\hbox{\char`\/}}}%
            \catcode`\.\active
            \catcode`\,\active
            \catcode`\;\active
            \catcode`\:\active
            \catcode`\?\active
            \catcode`\!\active
            \catcode`\/\active
            \lccode`\~`\~
        }
    \makeatother

    \let\OriginalVerbatim=\Verbatim
    \makeatletter
    \renewcommand{\Verbatim}[1][1]{%
        %\parskip\z@skip
        \sbox\Wrappedcontinuationbox {\Wrappedcontinuationsymbol}%
        \sbox\Wrappedvisiblespacebox {\FV@SetupFont\Wrappedvisiblespace}%
        \def\FancyVerbFormatLine ##1{\hsize\linewidth
            \vtop{\raggedright\hyphenpenalty\z@\exhyphenpenalty\z@
                \doublehyphendemerits\z@\finalhyphendemerits\z@
                \strut ##1\strut}%
        }%
        % If the linebreak is at a space, the latter will be displayed as visible
        % space at end of first line, and a continuation symbol starts next line.
        % Stretch/shrink are however usually zero for typewriter font.
        \def\FV@Space {%
            \nobreak\hskip\z@ plus\fontdimen3\font minus\fontdimen4\font
            \discretionary{\copy\Wrappedvisiblespacebox}{\Wrappedafterbreak}
            {\kern\fontdimen2\font}%
        }%

        % Allow breaks at special characters using \PYG... macros.
        \Wrappedbreaksatspecials
        % Breaks at punctuation characters . , ; ? ! and / need catcode=\active
        \OriginalVerbatim[#1,codes*=\Wrappedbreaksatpunct]%
    }
    \makeatother

    % Exact colors from NB
    \definecolor{incolor}{HTML}{303F9F}
    \definecolor{outcolor}{HTML}{D84315}
    \definecolor{cellborder}{HTML}{CFCFCF}
    \definecolor{cellbackground}{HTML}{F7F7F7}

    % prompt
    \makeatletter
    \newcommand{\boxspacing}{\kern\kvtcb@left@rule\kern\kvtcb@boxsep}
    \makeatother
    \newcommand{\prompt}[4]{
        {\ttfamily\llap{{\color{#2}[#3]:\hspace{3pt}#4}}\vspace{-\baselineskip}}
    }
    

    
    % Prevent overflowing lines due to hard-to-break entities
    \sloppy
    % Setup hyperref package
    \hypersetup{
      breaklinks=true,  % so long urls are correctly broken across lines
      colorlinks=true,
      urlcolor=urlcolor,
      linkcolor=linkcolor,
      citecolor=citecolor,
      }
    % Slightly bigger margins than the latex defaults
    
    \geometry{verbose,tmargin=1in,bmargin=1in,lmargin=1in,rmargin=1in}
    
\setcounter{tocdepth}{3}
\begin{document}
\begin{titlepage}
    \centering
  
    {\Huge\bfseries DeepFake Detection \\[1cm]} 
 
    {\Huge Fine-tuning Pretrained Networks to Identify the Best Model for Distinguishing Real and Fake Faces \\[2cm]} 
    
    \includegraphics[width=0.5\textwidth]{liotru.png} \\[1cm] 
    
 
    {\huge\textbf{Giuseppe Leonardi}} \\[0.5cm]
    {\Large 1000065630} \\[0.5cm]
    {\Large Data Science - UNICT} \\[1.5cm] 
    
    
    \vfill
\end{titlepage}


    \tableofcontents
    \newpage
\section{Introduction}\label{introduction}

    The goal of this project is to build and evaluate a system capable of
\textbf{distinguishing} between real and deep fake faces using
pre-trained neural networks. This system is intended to provide a robust
approach to identifying AI-generated images, which is crucial given the
growing prevalence of deep fake technologies.

Deep fake images and videos have raised concerns across various domains,
including cybersecurity, media authenticity, and personal privacy. The
development of reliable detection techniques is essential to mitigate
their misuse. This project explores the performance of pre-trained
models such as \emph{ResNet}, \emph{EfficientNet}, \emph{DenseNet} and
\emph{Vision Transformer} (ViT) for this task.

The dataset used in this project contains images of real faces and deep
fake faces, organized into two main categories:

\begin{enumerate}
\def\labelenumi{\arabic{enumi}.}
\item
  \textbf{Real Faces}: These images are stored in folders prefixed with
  ``0-''.
\item
  \textbf{DeepFake Faces}: These images are stored in folders prefixed
  with ``1-''.
\end{enumerate}

Each image represents a human face, with the deep fake images generated
using advanced artificial intelligence techniques such as StyleGAN or
StyleGAN2, which are widely used for generating synthetic facial images.
The real face images come from publicly available high-quality datasets
containing photographs of authentic human faces.

    After mounting Google Drive on Colab to seamlessly access the dataset
without overloading Colab's memory, I extracted the zip files containing
the images, organized based on their creation method. I structured the
dataset into two main folders: \emph{Real} and \emph{Fake}.

Before diving into data preprocessing, I explored an intriguing step:
transitioning these images from the spatial domain to the frequency
domain using the two-dimensional Fourier transform. This approach
allowed me to investigate whether images generated by certain models
exhibit recurring patterns or atypical structures.

To achieve this, I transformed each image into its frequency domain
representation, calculated the average frequency spectrum for each
folder, and then computed the overall averages for both Real and Deep
Fake images. This analysis serves as a foundation to uncover unique
traits within each category.

Further details will be discussed in the next section.

    \subsection{Fourier Spectra of the
Images}\label{fourier-spectra-of-the-images}

    In the context of analyzing images, the \textbf{spatial domain} alone
may not reveal subtle differences. Generative models like GANs often
produce images that visually resemble real ones but exhibit
\emph{distinctive patterns} in their underlying frequency distributions.

The Fourier Transform provides a way to examine these frequency
characteristics. By converting images from the spatial domain to the
\textbf{frequency domain}, we can uncover artifacts introduced by
generative models, differences in energy distribution, insights into the
generative process.

This makes Fourier analysis a powerful tool for understanding and
comparing the properties of real and fake images beyond their
appearance, making it an essential choice for this study.

The 2D Fourier Transform is a fundamental tool for analyzing the spatial
frequencies of an image. When applied to an image:

\begin{itemize}
\item
  \textbf{Low frequencies} (\emph{near the center of the spectrum}):
  correspond to slow variations (large structures or smooth gradients).
\item
  \textbf{High frequencies} (\emph{near the edges of the spectrum}):
  correspond to fine details or rapid variations (textures or sharp
  edges).
\end{itemize}

The result of the transform is an amplitude spectrum, representing the
energy of the spatial frequencies present in the image. In the
visualization the bright central region represents the dominance of low
frequencies (typical in natural images). Regular or symmetric patterns
indicate artifacts, often associated with synthetic images.

The Fourier Transform is particularly useful for analyzing differences
between real and artificially generated images. Generative models, such
as GANs (Generative Adversarial Networks), often introduce artifacts at
specific frequencies, which can be identified in the spectrum.

    \begin{tcolorbox}[breakable, size=fbox, boxrule=1pt, pad at break*=1mm,colback=cellbackground, colframe=cellborder]
\prompt{In}{incolor}{31}{\boxspacing}
\begin{Verbatim}[commandchars=\\\{\}]
\PY{n}{show\PYZus{}fourier}\PY{p}{(}\PY{p}{)}
\end{Verbatim}
\end{tcolorbox}
 
            
\prompt{Out}{outcolor}{31}{}
    
    \begin{center}
    \adjustimage{max size={0.8\linewidth}{0.8\paperheight}}{DeepFake_files/DeepFake_6_0.png}
    \end{center}
    { \hspace*{\fill} \\}
    

    Now, let's delve into the results obtained:

\begin{enumerate}
\def\labelenumi{\arabic{enumi}.}
\item
  \textbf{Real Images} (\emph{CelebA, FFHQ}): The spectra exhibit a
  \textbf{dominant central peak} with a gradual decay toward higher
  frequencies. These spectra are typical of natural images, where low
  frequencies dominate due to the spatial coherence of structures.
  \textbf{No periodic or symmetric patterns are observed}, indicating
  the absence of artificial artifacts in the data.
\item
  \textbf{Synthetic Images} (\emph{ATTGAN, GDWCT, StarGAN, STYLEGAN}):
\end{enumerate}

\begin{itemize}
\tightlist
\item
  ATTGAN and GDWCT: Show less pronounced \textbf{regular patterns} but
  still exhibit some periodic structures, indicating artifacts
  introduced by the generative process.
\item
  StarGAN and STYLEGAN: Display clear \textbf{grid-like patterns} and
  pronounced symmetries, with visible periodic frequencies.
\end{itemize}

These artifacts are characteristic of GAN models, which tend to
introduce unwanted frequencies during the image generation process.
These patterns may stem from the convolutional operations in the
generator.

For a more in-depth consultation, I saved the resulting images on Drive,
ensuring they are readily accessible

    \subsection{Preprocessing}\label{preprocessing}

    The preprocessing steps were carried out to ensure robustness and prevent overfitting. The preprocessing
steps include:

\begin{enumerate}
\def\labelenumi{\arabic{enumi}.}
\tightlist
\item
  \emph{Resizing}: The images were resized to a consistent size (224x224
  pixels), to ensure uniform processing.
\item
  \emph{Normalization}: To improve convergence during training, the
  images were normalized by a normalization standard to match the
  distribution of training data from pre-trained models, such as
  subtracting the mean and dividing by the standard deviation of the
  ImageNet dataset.
\item
  \emph{Augmentation}: To increase the variability of the images and
  enhance model robustness, augmentation techniques were applied,
  including rotation, translation, horizontal flipping, and brightness
  adjustments, to generate different versions of the images for
  training.
\end{enumerate}

The dataset was divided into three subsets:

\begin{itemize}
\tightlist
\item
  \emph{Training Set}: Used to train the deep learning models. About
  70\% of the images were allocated to training.
\item
  \emph{Validation Set}: Used to fine-tune model parameters and prevent
  overfitting. About 15\% of the images were allocated to validation.
\item
  \emph{Test Set}: Used to evaluate the model's generalization
  capability on unseen data. It will be used at the end of the studies.
\end{itemize}

This split allows for evaluating the model's effectiveness during and
after training, ensuring there is no bias and that the model can
generalize well to new images.

    \section{Methodology}\label{methodology}

    The following pre-trained models were selected for this project due to
their proven effectiveness in image classification tasks:

\texttt{DenseNet-121}: This model is known for its efficient
architecture that allows each layer to receive input from all previous
layers, facilitating better feature reuse and leading to improved
performance with fewer parameters. The number 121 refers to the number
of layers in the network, striking a balance between model complexity
and computational efficiency.

\texttt{ResNet-18}: ResNet (Residual Networks) are designed to combat
the vanishing gradient problem by using residual connections, allowing
for very deep networks without a significant loss in performance. The
``18'' refers to the number of layers in the network, which is the
smallest version of ResNet, making it ideal for tasks where
computational resources or training data are limited.

\texttt{EfficientNet-B0}: This model uses a novel compound scaling
method that scales depth, width, and resolution uniformly, providing
state-of-the-art accuracy with fewer parameters and lower computational
cost. EfficientNet has multiple variants (e.g., EfficientNet-B0, B1,
etc.), with the number indicating the model size. The larger the number,
the more computationally intensive and accurate the model becomes. For
this project, the B0 variant was selected based on the available
resources and required performance.

\texttt{Vision\ Transformer} (\texttt{ViT}): ViT leverages transformer
architectures, which have demonstrated remarkable performance in vision
tasks by processing image patches as sequences, similar to how text is
processed in NLP tasks. This model is known for its scalability and
performance on large datasets.

These models were selected for their ability to generalize well on image
classification tasks, as well as their diverse architectural
characteristics, which help explore different approaches to the problem.

    \section{Training Process}\label{training-process}

The training process was designed to ensure efficient model learning
while preventing overfitting and handling limited GPU memory. The
following key steps were implemented.

    \subsection{Training Setup}\label{training-setup}

    Loss Function (Criterion): The CrossEntropyLoss was used as the
criterion to evaluate the model's performance during training and
validation. This loss function is commonly used for multi-class
classification tasks, which aligns with the binary classification
problem of distinguishing real and deep fake faces.

Optimizer: The Adam optimizer was chosen due to its adaptive learning
rate capabilities, which help in faster convergence and stability during
training. The learning rate was set to 1e-4, a commonly used value for
fine-tuning pre-trained models.

Learning Rate Scheduler: The ReduceLROnPlateau scheduler was implemented
to reduce the learning rate if the validation loss plateaus, helping to
refine the model's training by providing smaller updates when necessary.
This helps avoid overshooting the optimal weights as the model
approaches convergence.

    \subsection{Training Loop}\label{training-loop}

    The train\_one\_epoch function handles the training for each epoch. For
each batch of images, the model performs the following steps:

Forward Pass: The input images are passed through the model to generate
predictions.

Loss Calculation: The model's predictions are compared with the true
labels using CrossEntropyLoss, which is computed for the batch.

Gradient Accumulation: To simulate larger batch sizes while maintaining
memory efficiency, the loss is divided by the accumulation\_steps (set
to 4 in this case). Gradients are accumulated over several mini-batches,
and the model's weights are updated only after accumulating gradients
for accumulation\_steps mini-batches.

Accuracy Tracking: The training accuracy is calculated by comparing the
model's predictions with the true labels. 

The validate\_one\_epoch\_with\_preds function is used for validating the
model at the end of each epoch:

Forward Pass: Like the training phase, the model generates predictions
for the validation dataset.

Loss and Accuracy Calculation: The loss and accuracy are computed for
the entire validation set.

Metrics Tracking: The true labels and predicted labels are stored to
calculate additional evaluation metrics (precision, recall, F1-score).

    \subsection{Evaluation Metrics}\label{evaluation-metrics}

    To monitor the model's performance during training, the following
metrics were computed:

Accuracy: Measures the percentage of correct predictions in both
training and validation sets.

Precision, Recall, and F1-Score: These metrics are calculated for the
validation set to assess the classifier's ability to correctly identify
real and deep fake faces, considering both false positives and false
negatives.

Confusion Matrix: The confusion matrix was computed at the end of each
epoch, providing a clear visualization of the model's classification
performance, including true positives, false positives, true negatives,
and false negatives.

    \subsection{Optimization Techniques}\label{optimization-techniques}

    To improve model training, the following techniques were employed:

Early Stopping: To prevent overfitting and unnecessary computation,
early stopping was implemented. If the validation loss does not improve
after a specified number of epochs (set to patience=3), training stops
early.

Batch Accumulation: Given the constraints on GPU memory, batch
accumulation was used to simulate larger batch sizes. This technique
allows for more effective weight updates without exceeding memory
limits, enabling the use of deeper and more complex models.

    \subsection{Model Evaluation and Saving
Results}\label{model-evaluation-and-saving-results}

    Best Model Selection: The model with the lowest validation loss was
saved as the best model. If the validation loss improved during an
epoch, the results (precision, recall, F1-score, and confusion matrix)
were updated.

Confusion Matrix Visualization: At the end of training, a heatmap of the
confusion matrix was generated, providing insights into the model's
classification performance across different classes (real vs.~fake
faces).

Results Storage: All the results (training and validation losses,
accuracies, and metrics for each epoch) were stored in a CSV file for
further analysis and visualization.

    \section{The Training}\label{the-training}

    Now, let's start to train each one of the choosen models, all of them
with all frozen layers, except for the last one, the Classifier.

    \subsection{ResNet-18}\label{resnet-18}

    \begin{tcolorbox}[breakable, size=fbox, boxrule=1pt, pad at break*=1mm,colback=cellbackground, colframe=cellborder]
\prompt{In}{incolor}{37}{\boxspacing}
\begin{Verbatim}[commandchars=\\\{\}]
\PY{n}{performance}\PY{p}{(}\PY{l+s+s2}{\PYZdq{}}\PY{l+s+s2}{ResNet\PYZhy{}18}\PY{l+s+s2}{\PYZdq{}}\PY{p}{)}
\end{Verbatim}
\end{tcolorbox}

    \begin{center}
    \adjustimage{max size={0.9\linewidth}{0.9\paperheight}}{DeepFake_files/DeepFake_26_0.png}
    \end{center}
    { \hspace*{\fill} \\}
    
    \begin{center}
    \adjustimage{max size={0.9\linewidth}{0.9\paperheight}}{DeepFake_files/DeepFake_26_1.png}
    \end{center}
    { \hspace*{\fill} \\}
    
    \begin{center}
    \adjustimage{max size={0.9\linewidth}{0.9\paperheight}}{DeepFake_files/DeepFake_26_2.png}
    \end{center}
    { \hspace*{\fill} \\}
    
    The model shows promising results overall, but there are areas where we
could see improvement.

The \textbf{training loss} is relatively low at \texttt{0.1441},
suggesting that the model has learned well from the training data.
However, the higher \textbf{validation loss} compared to the training
loss points to a potential issue of overfitting. This means that while
the model performs well on the training set, it might not generalize as
well to the validation set.

Looking at the training and validation accuracy, both values are fairly
close to each other, which is a positive sign, indicating that the model
is not significantly overfitting. The \textbf{training accuracy} of
\texttt{71.31\%} and \textbf{validation accuracy} of \texttt{72.25\%}
are both respectable, but there's still room to increase performance.

\textbf{Precision} stands at \texttt{68.85\%}, meaning that when the
model predicts a positive class, it is correct nearly 69\% of the time.
The \textbf{Recall} indicate that the model is identifying \texttt{70\%}
of the true positive cases. The F1 score is a combination of these two
metrics and reflects a good but improvable balance between them.

In terms of the \textbf{confusion matrix}, the model has correctly
classified \texttt{163\ true\ negatives} and
\texttt{126\ true\ positives}, which is good. However, there are some
errors: \texttt{57\ false\ positives} and \texttt{54\ false\ negatives}.
These errors could be minimized with model tuning, better feature
engineering, or even more training data.

    \subsection{EfficientNet}\label{efficientnet}

    \begin{tcolorbox}[breakable, size=fbox, boxrule=1pt, pad at break*=1mm,colback=cellbackground, colframe=cellborder]
\prompt{In}{incolor}{39}{\boxspacing}
\begin{Verbatim}[commandchars=\\\{\}]
\PY{n}{performance}\PY{p}{(}\PY{l+s+s2}{\PYZdq{}}\PY{l+s+s2}{EfficientNet}\PY{l+s+s2}{\PYZdq{}}\PY{p}{)}
\end{Verbatim}
\end{tcolorbox}

    \begin{center}
    \adjustimage{max size={0.9\linewidth}{0.9\paperheight}}{DeepFake_files/DeepFake_29_0.png}
    \end{center}
    { \hspace*{\fill} \\}
    
    \begin{center}
    \adjustimage{max size={0.9\linewidth}{0.9\paperheight}}{DeepFake_files/DeepFake_29_1.png}
    \end{center}
    { \hspace*{\fill} \\}
    
    \begin{center}
    \adjustimage{max size={0.9\linewidth}{0.9\paperheight}}{DeepFake_files/DeepFake_29_2.png}
    \end{center}
    { \hspace*{\fill} \\}
    
    The EfficientNet model demonstrates solid performance but leaves room
for improvement in some areas.

The \textbf{training loss} is relatively low at \texttt{0.1306},
indicating that the model has effectively learned patterns from the
training data. However, the higher \textbf{validation loss} of
\texttt{0.5223} suggests that the model may be struggling to generalize
to unseen data, which could be an indicator of mild overfitting.

When comparing the \textbf{training accuracy} of \texttt{80.87\%} with
the \textbf{validation accuracy} of \texttt{80.25\%}, the two values are
quite close. This consistency is a positive sign, as it implies that the
model is not significantly overfitting.

Looking deeper into the precision, recall, and F1 score:

\textbf{Precision} (\texttt{76.72\%}) shows that when the model predicts
a positive class, it is correct nearly 77\% of the time. \textbf{Recall}
(\texttt{80.56\%}) highlights that the model is successfully identifying
over 80\% of the true positive cases. The \textbf{F1 score}
(\texttt{78.59\%}) reflects a strong balance between precision and
recall, which is encouraging.

From the confusion matrix, we can see that the model correctly
classified:

\begin{enumerate}
\def\labelenumi{\arabic{enumi}.}
\tightlist
\item
  176 true negatives;
\item
  145 true positives;
\end{enumerate}

However, there are notable errors that suggest potential areas for
improvement. For instance, reducing false negatives would boost recall,
while minimizing false positives would improve precision.

Overall, EfficientNet shows promise with good generalization, but there
is still scope for optimization to achieve higher performance.

    \subsection{DenseNet-121}\label{densenet-121}

    \begin{tcolorbox}[breakable, size=fbox, boxrule=1pt, pad at break*=1mm,colback=cellbackground, colframe=cellborder]
\prompt{In}{incolor}{41}{\boxspacing}
\begin{Verbatim}[commandchars=\\\{\}]
\PY{n}{performance}\PY{p}{(}\PY{l+s+s2}{\PYZdq{}}\PY{l+s+s2}{DenseNet121}\PY{l+s+s2}{\PYZdq{}}\PY{p}{)}
\end{Verbatim}
\end{tcolorbox}

    \begin{center}
    \adjustimage{max size={0.9\linewidth}{0.9\paperheight}}{DeepFake_files/DeepFake_32_0.png}
    \end{center}
    { \hspace*{\fill} \\}
    
    \begin{center}
    \adjustimage{max size={0.9\linewidth}{0.9\paperheight}}{DeepFake_files/DeepFake_32_1.png}
    \end{center}
    { \hspace*{\fill} \\}
    
    \begin{center}
    \adjustimage{max size={0.9\linewidth}{0.9\paperheight}}{DeepFake_files/DeepFake_32_2.png}
    \end{center}
    { \hspace*{\fill} \\}
    
    The \textbf{training loss} is relatively low at \texttt{0.1325}, which
indicates that the model has successfully learned patterns from the
training data. However, the \textbf{validation loss} of \texttt{0.5293}
is higher, suggesting that the model might be struggling to generalize
to new, unseen data. This discrepancy between the training and
validation losses could point to mild overfitting.

When comparing the \textbf{training accuracy} of \texttt{79.75\%} and
the validation accuracy of \texttt{80.75\%}, the close proximity between
the two values suggests that the model is not severely overfitting. This
is a positive sign, as it indicates that the model performs similarly on
both the training and validation sets.

Looking at precision, recall, and F1 score:

\textbf{Precision} (\texttt{78.30\%}) tells us that when the model
predicts a positive class (fake face), it is correct \texttt{78.30\%} of
the time. This is a good result, but there is still room for improvement
to reduce false positives (real faces misclassified as fake).

\textbf{Recall} (\texttt{84.26\%}) indicates that the model successfully
identifies 84.26\% of the true positives (fake faces). This is a strong
result, but further improvement could be made by reducing false
negatives (fake faces misclassified as real).

The \textbf{F1 score} (\texttt{81.17\%}) reflects a strong balance
between precision and recall, indicating that the model has achieved a
solid trade-off between these two metrics.

In summary, DenseNet-121 shows good generalization with a strong ability
to identify fake faces. However, there is still potential for
improvement, particularly in reducing misclassifications, such as false
positives and false negatives, to boost both precision and recall.
Further optimization can lead to better performance.

    \subsection{ViT}\label{vit}

    \begin{tcolorbox}[breakable, size=fbox, boxrule=1pt, pad at break*=1mm,colback=cellbackground, colframe=cellborder]
\prompt{In}{incolor}{43}{\boxspacing}
\begin{Verbatim}[commandchars=\\\{\}]
\PY{n}{performance}\PY{p}{(}\PY{l+s+s2}{\PYZdq{}}\PY{l+s+s2}{Vit}\PY{l+s+s2}{\PYZdq{}}\PY{p}{)}
\end{Verbatim}
\end{tcolorbox}

    \begin{center}
    \adjustimage{max size={0.9\linewidth}{0.9\paperheight}}{DeepFake_files/DeepFake_35_0.png}
    \end{center}
    { \hspace*{\fill} \\}
    
    \begin{center}
    \adjustimage{max size={0.9\linewidth}{0.9\paperheight}}{DeepFake_files/DeepFake_35_1.png}
    \end{center}
    { \hspace*{\fill} \\}
    
    \begin{center}
    \adjustimage{max size={0.9\linewidth}{0.9\paperheight}}{DeepFake_files/DeepFake_35_2.png}
    \end{center}
    { \hspace*{\fill} \\}
    
    The ViT (Vision Transformer) model shows promising results with room for
further improvement in some areas.

The \textbf{training loss} starts at \texttt{0.1851} and decreases to
\texttt{0.1231} by epoch 20, suggesting that the model effectively
learns patterns from the training data. However, the** validation loss**
remains higher, starting at \texttt{0.7129} and dropping to
\texttt{0.5204} by epoch 20. While the model's validation loss decreases
over time, it's still higher than the training loss, indicating some
level of overfitting. The model may be performing well on the training
data but struggling to generalize to unseen data.

When we look at the \textbf{training accuracy}, it increases from
\texttt{51.56\%} in the first epoch to \texttt{76.81\%} in the final
epoch, demonstrating an improvement in the model's ability to correctly
classify the training set. Similarly, the \textbf{validation accuracy}
starts at \texttt{55\%} and rises to \texttt{75.25\%} by epoch 20,
showing a consistent improvement.

\textbf{Precision} indicates that when the model predicts a positive
class (fake face), it is correct \texttt{75\%} of the time. The
\textbf{Recall} highlights that the model identifies \texttt{74.62\%} of
the true positives (fake faces). Although recall is relatively high,
further improvements could still reduce the number of false negatives
(fake faces misclassified as real). \textbf{F1 score} (\texttt{74.81\%})
represents a balanced performance between precision and recall,
indicating the model's strength in both correctly identifying fake faces
and minimizing misclassifications.

Looking at the \textbf{confusion matrix}, The presence of errors
suggests areas where the model could be further optimized, specifically
by reducing false positives to improve precision and false negatives to
enhance recall.

In summary, the ViT model performs well with solid generalization across
training and validation sets. While precision, recall, and F1 score are
strong, there is still potential to reduce misclassifications,
especially false positives and false negatives, through further tuning
and optimization.

    \subsection{The Choice}\label{the-choice}

    Now, compare each model to others to choose of them:

    \begin{tcolorbox}[breakable, size=fbox, boxrule=1pt, pad at break*=1mm,colback=cellbackground, colframe=cellborder]
\prompt{In}{incolor}{44}{\boxspacing}
\begin{Verbatim}[commandchars=\\\{\}]
\PY{n}{model\PYZus{}names} \PY{o}{=} \PY{p}{[}\PY{l+s+s1}{\PYZsq{}}\PY{l+s+s1}{ResNet\PYZhy{}18}\PY{l+s+s1}{\PYZsq{}}\PY{p}{,} \PY{l+s+s1}{\PYZsq{}}\PY{l+s+s1}{EfficientNet}\PY{l+s+s1}{\PYZsq{}}\PY{p}{,} \PY{l+s+s1}{\PYZsq{}}\PY{l+s+s1}{DenseNet121}\PY{l+s+s1}{\PYZsq{}}\PY{p}{,} \PY{l+s+s1}{\PYZsq{}}\PY{l+s+s1}{Vit}\PY{l+s+s1}{\PYZsq{}}\PY{p}{]}
\PY{n}{recap}\PY{p}{(}\PY{n}{model\PYZus{}names}\PY{p}{)}
\end{Verbatim}
\end{tcolorbox}

    \begin{center}
    \adjustimage{max size={0.9\linewidth}{0.9\paperheight}}{DeepFake_files/DeepFake_39_0.png}
    \end{center}
    { \hspace*{\fill} \\}
    
    \textbf{\emph{EfficientNet}} stands out as an excellent choice,
combining high performance with lightweight efficiency. With a
\textbf{validation accuracy} of \texttt{80.25\%} and an \textbf{F1
score} of \texttt{78.59\%}, it demonstrates an excellent balance between
precision and recall while remaining a lighter model, ideal for
resource-constrained applications.

Its slightly higher \textbf{precision} (\texttt{76.72\%}) compared to
\textbf{DenseNet121} (\texttt{78.30\%}) makes it particularly suitable
for reducing false positives, which is critical in tasks like detecting
fake faces where minimizing errors is essential. Additionally, the
lowest \textbf{validation loss} (\texttt{0.5223}) among the models
indicates that \textbf{EfficientNet} generalizes better on the
validation dataset.

While \textbf{DenseNet121} offers a slight edge in recall
(\texttt{84.26\%} vs.~\texttt{80.56\%}), its higher computational cost
makes it less practical for environments with limited hardware resources
or where speed is a key factor. Compared to \textbf{ResNet-18} and
\textbf{ViT}, \textbf{\emph{EfficientNet}} clearly outperforms both in
accuracy and F1 score, making it the ideal compromise between
performance and efficiency.

In summary, \textbf{\emph{EfficientNet}} represents the most balanced
solution when looking for an efficient yet high-performing model. For
applications requiring a lightweight implementation without sacrificing
too much in terms of accuracy and overall performance, EfficientNet is
the best choice.

    \section{EfficientNet\_2 (After Unlocking
Layers)}\label{efficientnet_2-after-unlocking-layers}

    Since EfficientNet has demonstrated commendable results when applied to
our dataset, the next logical step is to unfreeze additional layers
preceding the classifier. This approach allows us to fine-tune the
weights in these layers, which could potentially lead to improved
performance. Specifically, I have unfrozen the last two layers of the
features block, which is responsible for extracting high-level
representations from the input data through a series of convolutional
and pooling operations. By fine-tuning these layers, we aim to enhance
the model's ability to capture more nuanced patterns in the data.

I would also like to highlight that, while a scheduler has been declared
in the function, it is not yet being used. Its inclusion was a strategic
choice for scalability, ensuring a smoother implementation in future
training phases if needed.

Now, let's start with the traing and see the results.

    \begin{tcolorbox}[breakable, size=fbox, boxrule=1pt, pad at break*=1mm,colback=cellbackground, colframe=cellborder]
\prompt{In}{incolor}{45}{\boxspacing}
\begin{Verbatim}[commandchars=\\\{\}]
\PY{n}{effnet2} \PY{o}{=} \PY{n}{efficientnet\PYZus{}b0}\PY{p}{(}\PY{n}{weights}\PY{o}{=}\PY{n}{EfficientNet\PYZus{}B0\PYZus{}Weights}\PY{o}{.}\PY{n}{DEFAULT}\PY{p}{)}
\PY{n}{effnet2}\PY{o}{.}\PY{n}{classifier} \PY{o}{=} \PY{n}{nn}\PY{o}{.}\PY{n}{Linear}\PY{p}{(}\PY{n}{effnet2}\PY{o}{.}\PY{n}{classifier}\PY{p}{[}\PY{l+m+mi}{1}\PY{p}{]}\PY{o}{.}\PY{n}{in\PYZus{}features}\PY{p}{,} \PY{l+m+mi}{2}\PY{p}{)}

\PY{k}{for} \PY{n}{param} \PY{o+ow}{in} \PY{n}{effnet2}\PY{o}{.}\PY{n}{parameters}\PY{p}{(}\PY{p}{)}\PY{p}{:}
    \PY{n}{param}\PY{o}{.}\PY{n}{requires\PYZus{}grad} \PY{o}{=} \PY{k+kc}{False}

\PY{k}{for} \PY{n}{param} \PY{o+ow}{in} \PY{n}{effnet2}\PY{o}{.}\PY{n}{classifier}\PY{o}{.}\PY{n}{parameters}\PY{p}{(}\PY{p}{)}\PY{p}{:}
    \PY{n}{param}\PY{o}{.}\PY{n}{requires\PYZus{}grad} \PY{o}{=} \PY{k+kc}{True}

\PY{k}{for} \PY{n}{layer} \PY{o+ow}{in} \PY{n}{effnet2}\PY{o}{.}\PY{n}{features}\PY{p}{[}\PY{o}{\PYZhy{}}\PY{l+m+mi}{2}\PY{p}{:}\PY{p}{]}\PY{p}{:}
    \PY{k}{for} \PY{n}{param} \PY{o+ow}{in} \PY{n}{layer}\PY{o}{.}\PY{n}{parameters}\PY{p}{(}\PY{p}{)}\PY{p}{:}
        \PY{n}{param}\PY{o}{.}\PY{n}{requires\PYZus{}grad} \PY{o}{=} \PY{k+kc}{True}

\PY{c+c1}{\PYZsh{}Execute Training}
\PY{n}{criterion}\PY{p}{,} \PY{n}{optimizer}\PY{p}{,} \PY{n}{scheduler} \PY{o}{=} \PY{n}{setup\PYZus{}training}\PY{p}{(}\PY{n}{effnet2}\PY{p}{)}

\PY{n}{effnet\PYZus{}2Train} \PY{o}{=} \PY{n}{train\PYZus{}model}\PY{p}{(}\PY{n}{effnet2}\PY{p}{,} \PY{l+s+s2}{\PYZdq{}}\PY{l+s+s2}{EfficientNet\PYZus{}2}\PY{l+s+s2}{\PYZdq{}}\PY{p}{,} \PY{n}{criterion}\PY{p}{,} \PY{n}{optimizer}\PY{p}{,} \PY{n}{train\PYZus{}loader}\PY{o}{=}\PY{n}{train\PYZus{}loader}\PY{p}{,} \PY{n}{val\PYZus{}loader}\PY{o}{=}\PY{n}{val\PYZus{}loader}\PY{p}{)}
\end{Verbatim}
\end{tcolorbox}

    \begin{Verbatim}[commandchars=\\\{\}]

Model 'EfficientNet\_2' already exists in
/content/drive/MyDrive/deepfake\_dataset/models/EfficientNet\_2. Training skipped.
    \end{Verbatim}

    \begin{tcolorbox}[breakable, size=fbox, boxrule=1pt, pad at break*=1mm,colback=cellbackground, colframe=cellborder]
\prompt{In}{incolor}{46}{\boxspacing}
\begin{Verbatim}[commandchars=\\\{\}]
\PY{n}{performance}\PY{p}{(}\PY{l+s+s2}{\PYZdq{}}\PY{l+s+s2}{EfficientNet\PYZus{}2}\PY{l+s+s2}{\PYZdq{}}\PY{p}{)}
\end{Verbatim}
\end{tcolorbox}

    \begin{center}
    \adjustimage{max size={0.9\linewidth}{0.9\paperheight}}{DeepFake_files/DeepFake_44_0.png}
    \end{center}
    { \hspace*{\fill} \\}
    
    \begin{center}
    \adjustimage{max size={0.9\linewidth}{0.9\paperheight}}{DeepFake_files/DeepFake_44_1.png}
    \end{center}
    { \hspace*{\fill} \\}
    
    \begin{center}
    \adjustimage{max size={0.9\linewidth}{0.9\paperheight}}{DeepFake_files/DeepFake_44_2.png}
    \end{center}
    { \hspace*{\fill} \\}
    
    After unlocking two additional levels of EfficientNet, the model's
performance has shown significant improvement, demonstrating its
capacity to generalize better and deliver more accurate predictions.
Let's analyze the updated results in detail.

The \textbf{training loss} has dropped to an impressive \texttt{0.0166},
while the \textbf{validation loss} has decreased to \texttt{0.1596}.
This reduction from the initial training suggests that the extended
training with additional layers has allowed the model to better learn
from the data while maintaining a strong ability to generalize. The gap
between training and validation loss is minimal, indicating that the
overfitting observed earlier has been effectively mitigated.

The \textbf{training accuracy} has reached a near-perfect level at
\texttt{98.5\%}, which is an excellent result. The validation accuracy,
now at \texttt{92.5\%}, reflects significant improvement compared to the
previous \texttt{80.25\%}. The small gap between training and validation
accuracy highlights the model's robustness and ability to generalize
effectively.

This leap in accuracy signifies that the model is now capable of
distinguishing between real and fake faces with greater precision. The
higher capacity layers in EfficientNet have likely enabled the model to
capture subtle patterns and variations in the dataset.

The \textbf{F1 score}, a balanced metric, now stands at
\texttt{92.57\%}, up from \texttt{78.59\%}. This improvement highlights
a better equilibrium between precision and recall, demonstrating overall
reliability and performance. The \textbf{confusion matrix} reflects the
strides made by the model and the errors have been significantly
minimized compared to the initial results, suggesting that the
additional layers have enhanced the model's feature extraction and
discrimination capabilities.

While the results are highly promising, \emph{there is always room for
further optimization}: in fact, the next step is to try
``\textbf{Fine-Tuning Hyperparameters}'' and to be precise I wanna use a
scheduler to adapt the learning rate during the training.

So, EfficientNet has demonstrated its power, and with these unlocked
levels, it has set a solid benchmark for the task of distinguishing real
from fake faces. This progress is highly encouraging, and further
fine-tuning may lead to near-perfect results.

    \section{EfficientNet\_3 (With Plateau
Scheduler)}\label{efficientnet_3-with-plateau-scheduler}

    A Scheduler is a method to adjust the learning rate during training.

Schedulers are used to speed up training and improve convergence,
adapting the learning rate in various ways, either based on the epoch
count or based on the model's performance, like the one i've choose for
this case.

I'm using \textbf{ReduceLROnPlateau}, a dynamic learning rate scheduler
in PyTorch that adjusts the learning rate based on the performance of a
monitored metric. The scheduler is linked to the optimizer and monitores
a metric that should decrease (the \textbf{validation loss}). If the
metric stops decreasing, it reduces the learning rate.

    \begin{tcolorbox}[breakable, size=fbox, boxrule=1pt, pad at break*=1mm,colback=cellbackground, colframe=cellborder]
\prompt{In}{incolor}{47}{\boxspacing}
\begin{Verbatim}[commandchars=\\\{\}]
\PY{n}{effnet3} \PY{o}{=} \PY{n}{efficientnet\PYZus{}b0}\PY{p}{(}\PY{n}{weights}\PY{o}{=}\PY{n}{EfficientNet\PYZus{}B0\PYZus{}Weights}\PY{o}{.}\PY{n}{DEFAULT}\PY{p}{)}
\PY{n}{effnet3}\PY{o}{.}\PY{n}{classifier} \PY{o}{=} \PY{n}{nn}\PY{o}{.}\PY{n}{Linear}\PY{p}{(}\PY{n}{effnet3}\PY{o}{.}\PY{n}{classifier}\PY{p}{[}\PY{l+m+mi}{1}\PY{p}{]}\PY{o}{.}\PY{n}{in\PYZus{}features}\PY{p}{,} \PY{l+m+mi}{2}\PY{p}{)}

\PY{k}{for} \PY{n}{param} \PY{o+ow}{in} \PY{n}{effnet3}\PY{o}{.}\PY{n}{parameters}\PY{p}{(}\PY{p}{)}\PY{p}{:}
    \PY{n}{param}\PY{o}{.}\PY{n}{requires\PYZus{}grad} \PY{o}{=} \PY{k+kc}{False}

\PY{k}{for} \PY{n}{param} \PY{o+ow}{in} \PY{n}{effnet3}\PY{o}{.}\PY{n}{classifier}\PY{o}{.}\PY{n}{parameters}\PY{p}{(}\PY{p}{)}\PY{p}{:}
    \PY{n}{param}\PY{o}{.}\PY{n}{requires\PYZus{}grad} \PY{o}{=} \PY{k+kc}{True}

\PY{k}{for} \PY{n}{layer} \PY{o+ow}{in} \PY{n}{effnet3}\PY{o}{.}\PY{n}{features}\PY{p}{[}\PY{o}{\PYZhy{}}\PY{l+m+mi}{2}\PY{p}{:}\PY{p}{]}\PY{p}{:}
    \PY{k}{for} \PY{n}{param} \PY{o+ow}{in} \PY{n}{layer}\PY{o}{.}\PY{n}{parameters}\PY{p}{(}\PY{p}{)}\PY{p}{:}
        \PY{n}{param}\PY{o}{.}\PY{n}{requires\PYZus{}grad} \PY{o}{=} \PY{k+kc}{True}

\PY{c+c1}{\PYZsh{}Execute Training}
\PY{n}{criterion}\PY{p}{,} \PY{n}{optimizer}\PY{p}{,} \PY{n}{scheduler} \PY{o}{=} \PY{n}{setup\PYZus{}training}\PY{p}{(}\PY{n}{effnet3}\PY{p}{)}

\PY{n}{effnet\PYZus{}3Train} \PY{o}{=} \PY{n}{train\PYZus{}model}\PY{p}{(}\PY{n}{effnet3}\PY{p}{,} \PY{l+s+s2}{\PYZdq{}}\PY{l+s+s2}{EfficientNet\PYZus{}3}\PY{l+s+s2}{\PYZdq{}}\PY{p}{,} \PY{n}{criterion}\PY{p}{,} \PY{n}{optimizer}\PY{p}{,} \PY{n}{train\PYZus{}loader}\PY{o}{=}\PY{n}{train\PYZus{}loader}\PY{p}{,} \PY{n}{val\PYZus{}loader}\PY{o}{=}\PY{n}{val\PYZus{}loader}\PY{p}{)}
\end{Verbatim}
\end{tcolorbox}

    \begin{Verbatim}[commandchars=\\\{\}]

Model 'EfficientNet\_3' already exists in
/content/drive/MyDrive/deepfake\_dataset/models/EfficientNet\_3. Training skipped.
    \end{Verbatim}

    \begin{tcolorbox}[breakable, size=fbox, boxrule=1pt, pad at break*=1mm,colback=cellbackground, colframe=cellborder]
\prompt{In}{incolor}{48}{\boxspacing}
\begin{Verbatim}[commandchars=\\\{\}]
\PY{n}{performance}\PY{p}{(}\PY{l+s+s2}{\PYZdq{}}\PY{l+s+s2}{EfficientNet\PYZus{}3}\PY{l+s+s2}{\PYZdq{}}\PY{p}{)}
\end{Verbatim}
\end{tcolorbox}

    \begin{center}
    \adjustimage{max size={0.9\linewidth}{0.9\paperheight}}{DeepFake_files/DeepFake_49_0.png}
    \end{center}
    { \hspace*{\fill} \\}
    
    \begin{center}
    \adjustimage{max size={0.9\linewidth}{0.9\paperheight}}{DeepFake_files/DeepFake_49_1.png}
    \end{center}
    { \hspace*{\fill} \\}
    
    \begin{center}
    \adjustimage{max size={0.9\linewidth}{0.9\paperheight}}{DeepFake_files/DeepFake_49_2.png}
    \end{center}
    { \hspace*{\fill} \\}
    
    The introduction of the Reduce on Plateau scheduler doesn't appears to have been
a beneficial adjustment. While the training accuracy (98.12\%) is
slightly lower than that of EfficientNet\_2 (98.5\%), the validation
accuracy has improved to 93\%, alongside an increase in the F1 Score to
0.9364. This suggests that the scheduler helped the model generalize
better by dynamically adjusting the learning rate, particularly during
stagnation in performance.

The most notable improvement is in the Recall, which increased
significantly from 0.935 to 0.9671. This indicates that the model is now
more effective at correctly identifying positive cases, a critical
improvement depending on the task's objectives.

While the differences in metrics are subtle, this version demonstrates
improved robustness and stability, making it a strong candidate if
minimizing overfitting and maximizing positive detection is a priority.
However, the choice between this version and EfficientNet\_2 ultimately
depends on whether the incremental gains justify the added complexity of
using the scheduler.

    \section{Predictions}\label{predictions}

    \begin{tcolorbox}[breakable, size=fbox, boxrule=1pt, pad at break*=1mm,colback=cellbackground, colframe=cellborder]
\prompt{In}{incolor}{49}{\boxspacing}
\begin{Verbatim}[commandchars=\\\{\}]
\PY{n}{model\PYZus{}names} \PY{o}{=} \PY{p}{[}\PY{l+s+s1}{\PYZsq{}}\PY{l+s+s1}{EfficientNet}\PY{l+s+s1}{\PYZsq{}}\PY{p}{,} \PY{l+s+s1}{\PYZsq{}}\PY{l+s+s1}{EfficientNet\PYZus{}2}\PY{l+s+s1}{\PYZsq{}}\PY{p}{,} \PY{l+s+s1}{\PYZsq{}}\PY{l+s+s1}{EfficientNet\PYZus{}3}\PY{l+s+s1}{\PYZsq{}}\PY{p}{]}
\PY{n}{recap}\PY{p}{(}\PY{n}{model\PYZus{}names}\PY{p}{)}
\end{Verbatim}
\end{tcolorbox}

    \begin{center}
    \adjustimage{max size={0.9\linewidth}{0.9\paperheight}}{DeepFake_files/DeepFake_52_0.png}
    \end{center}
    { \hspace*{\fill} \\}
    
    Both \emph{EfficientNet\_2} and \emph{EfficientNet\_3} show substantial
improvements over the base model, with the additional layers and the plateau scheduler that have
clearly enhanced the model's learning ability, resulting in:

\begin{enumerate}
\def\labelenumi{\arabic{enumi}.}
\tightlist
\item
  A substantial reduction in both training and validation loss.
\item
  A significant increase in training and validation accuracy.
\item
  Stronger precision, recall, and F1 score, indicating that the model is
  more reliable in distinguishing between real and fake faces.
\end{enumerate}

These improvements set EfficientNet\_3 as the best version for the task,
with its ability to generalize effectively and deliver high-quality
predictions. Fine-tuning and further optimizations could push
performance even further, but these results are already highly
promising.

What we will do now is test the behavior of the EfficientNet\_3 model on
the test set, which consists of a series of images whose origin (and
therefore their classification) is unknown to us. This will allow us to
evaluate how well our model performs when faced with completely unseen
data.

    \subsection{Test Set}\label{test-set}

    Now, all that's left is to load the test set, load the model weights and
structure, and finally start the training. For simplicity, the first six
images and their classifications will be displayed. For more details,
please refer to the CSV file containing all the predictions.

    \begin{tcolorbox}[breakable, size=fbox, boxrule=1pt, pad at break*=1mm,colback=cellbackground, colframe=cellborder]
\prompt{In}{incolor}{70}{\boxspacing}
\begin{Verbatim}[commandchars=\\\{\}]
\PY{n}{classified}\PY{p}{(}\PY{p}{)}
\end{Verbatim}
\end{tcolorbox}

    \begin{center}
    \adjustimage{max size={0.9\linewidth}{0.9\paperheight}}{DeepFake_files/DeepFake_56_0.png}
    \end{center}
    { \hspace*{\fill} \\}
    
    \section{Conclusions}\label{conclusions}

    The results obtained using the EfficientNet\_3 model on the test set
demonstrate promising performance, with strong indicators of its ability
to distinguish between real and fake faces.

    \begin{tcolorbox}[breakable, size=fbox, boxrule=1pt, pad at break*=1mm,colback=cellbackground, colframe=cellborder]
\prompt{In}{incolor}{76}{\boxspacing}
\begin{Verbatim}[commandchars=\\\{\}]
\PY{n}{final\PYZus{}result}\PY{p}{(}\PY{p}{)}
\end{Verbatim}
\end{tcolorbox}

    \begin{center}
    \adjustimage{max size={0.9\linewidth}{0.9\paperheight}}{DeepFake_files/DeepFake_59_0.png}
    \end{center}
    { \hspace*{\fill} \\}
    
    \begin{enumerate}
\def\labelenumi{\arabic{enumi}.}
\item
  \textbf{Accuracy} (\texttt{0.80}): The model achieved an impressive
  accuracy of 80\%, correctly classifying a large proportion of the test
  set. This shows the model's general effectiveness in distinguishing
  real from fake faces, ensuring high reliability in the results.
\item
  \textbf{Precision} (\texttt{0.8613}): The model achieved a high
  precision of 80\%, meaning that 80\% of the predicted fake faces were
  indeed fake. This is a positive outcome, demonstrating the model's
  efficiency in minimizing false positives, where real faces are
  misclassified as fake.
\item
  \textbf{Recall} (\texttt{0.8514}): With a recall of 76.57\%, the model
  effectively identified the majority of the fake faces. While there is
  still a small margin for improvement, the model's ability to capture
  fake faces demonstrates a solid understanding of the problem, ensuring
  minimal missed detections.
\item
  \textbf{F1-score} (\texttt{0.8563}): The F1-score of 0.7637 reflects a
  well-balanced performance between precision and recall. This indicates
  that the model maintains an excellent trade-off between these two
  metrics, providing a robust overall classification capability.
\end{enumerate}

In terms of \textbf{\emph{future improvements}}, there are several
directions that could further enhance the performance of the
EfficientNet\_3 model. One promising approach is to continue the
\emph{model fine-tuning}. Additionally, the application of
\emph{ensemble methods} could offer another level of performance. By
combining multiple models, each contributing its unique strengths, the
model's overall classification capability could be enhanced, leading to
more accurate predictions.

In conclusion, while the EfficientNet\_3 model has already shown
promising results, there is still considerable potential for
improvement. By focusing on these targeted strategies, the model's
performance could be further optimized, making it an even more powerful
tool for DeepFake detection in the future.

    \section{Created Functions}\label{created-functions}

In this section, you will find all the functions personally created by
me to avoid repeating lines of code endlessly and to simplify the
workflow.

ATTENTION: MATERIAL RESERVED FOR AUTHORIZED PERSONNEL ONLY.

    \subsection{Fourier}\label{fourier}

I developed a function that iterates through each folder, converting
each image into its frequency domain using NumPy's
\texttt{np.fft.fft2()} function. Then, with \texttt{np.mean()} Codice in
linea, I calculated the average of these image spectra and computed the
final averages for both real and fake images.

    \begin{tcolorbox}[breakable, size=fbox, boxrule=1pt, pad at break*=1mm,colback=cellbackground, colframe=cellborder]
\prompt{In}{incolor}{4}{\boxspacing}
\begin{Verbatim}[commandchars=\\\{\}]
\PY{k}{def} \PY{n+nf}{mean\PYZus{}fourier}\PY{p}{(}\PY{n}{image\PYZus{}folder}\PY{p}{,} \PY{n}{image\PYZus{}format}\PY{o}{=}\PY{l+s+s2}{\PYZdq{}}\PY{l+s+s2}{*.png}\PY{l+s+s2}{\PYZdq{}}\PY{p}{,} \PY{n}{resize\PYZus{}dim}\PY{o}{=}\PY{k+kc}{None}\PY{p}{)}\PY{p}{:}

    \PY{n}{image\PYZus{}paths} \PY{o}{=} \PY{n}{glob}\PY{o}{.}\PY{n}{glob}\PY{p}{(}\PY{l+s+sa}{f}\PY{l+s+s2}{\PYZdq{}}\PY{l+s+si}{\PYZob{}}\PY{n}{image\PYZus{}folder}\PY{l+s+si}{\PYZcb{}}\PY{l+s+s2}{/}\PY{l+s+si}{\PYZob{}}\PY{n}{image\PYZus{}format}\PY{l+s+si}{\PYZcb{}}\PY{l+s+s2}{\PYZdq{}}\PY{p}{)}
    \PY{n}{specter} \PY{o}{=} \PY{p}{[}\PY{p}{]}

    \PY{k}{for} \PY{n}{path} \PY{o+ow}{in} \PY{n}{image\PYZus{}paths}\PY{p}{:}
        \PY{n}{img} \PY{o}{=} \PY{n}{cv2}\PY{o}{.}\PY{n}{imread}\PY{p}{(}\PY{n}{path}\PY{p}{,} \PY{n}{cv2}\PY{o}{.}\PY{n}{IMREAD\PYZus{}GRAYSCALE}\PY{p}{)}
        \PY{k}{if} \PY{n}{img} \PY{o+ow}{is} \PY{k+kc}{None}\PY{p}{:}
            \PY{k}{continue}  \PY{c+c1}{\PYZsh{} skip if image is none}
        \PY{c+c1}{\PYZsh{} Resize the image}
        \PY{k}{if} \PY{n}{resize\PYZus{}dim} \PY{o+ow}{is} \PY{o+ow}{not} \PY{k+kc}{None}\PY{p}{:}
            \PY{n}{img} \PY{o}{=} \PY{n}{cv2}\PY{o}{.}\PY{n}{resize}\PY{p}{(}\PY{n}{img}\PY{p}{,} \PY{n}{resize\PYZus{}dim}\PY{p}{)}
        \PY{c+c1}{\PYZsh{} Apply the Fourier Transform}
        \PY{n}{f\PYZus{}transform} \PY{o}{=} \PY{n}{np}\PY{o}{.}\PY{n}{fft}\PY{o}{.}\PY{n}{fft2}\PY{p}{(}\PY{n}{img}\PY{p}{)}
        \PY{n}{f\PYZus{}shift} \PY{o}{=} \PY{n}{np}\PY{o}{.}\PY{n}{fft}\PY{o}{.}\PY{n}{fftshift}\PY{p}{(}\PY{n}{f\PYZus{}transform}\PY{p}{)}  \PY{c+c1}{\PYZsh{} Shift to center}
        \PY{n}{magnitude\PYZus{}spectrum} \PY{o}{=} \PY{n}{np}\PY{o}{.}\PY{n}{log}\PY{p}{(}\PY{l+m+mi}{1} \PY{o}{+} \PY{n}{np}\PY{o}{.}\PY{n}{abs}\PY{p}{(}\PY{n}{f\PYZus{}shift}\PY{p}{)}\PY{p}{)} \PY{c+c1}{\PYZsh{}log to simply the visualization}
        \PY{n}{specter}\PY{o}{.}\PY{n}{append}\PY{p}{(}\PY{n}{magnitude\PYZus{}spectrum}\PY{p}{)}

    \PY{k}{if} \PY{o+ow}{not} \PY{n}{specter}\PY{p}{:}
        \PY{k}{raise} \PY{n+ne}{ValueError}\PY{p}{(}\PY{l+s+s2}{\PYZdq{}}\PY{l+s+s2}{No valid Image to transform!}\PY{l+s+s2}{\PYZdq{}}\PY{p}{)}

    \PY{c+c1}{\PYZsh{}return the mean of each folder}
    \PY{n}{mean\PYZus{}specter} \PY{o}{=} \PY{n}{np}\PY{o}{.}\PY{n}{mean}\PY{p}{(}\PY{n}{specter}\PY{p}{,} \PY{n}{axis}\PY{o}{=}\PY{l+m+mi}{0}\PY{p}{)}
    \PY{k}{return} \PY{n}{mean\PYZus{}specter}
\end{Verbatim}
\end{tcolorbox}

    \begin{tcolorbox}[breakable, size=fbox, boxrule=1pt, pad at break*=1mm,colback=cellbackground, colframe=cellborder]
\prompt{In}{incolor}{30}{\boxspacing}
\begin{Verbatim}[commandchars=\\\{\}]
\PY{k}{def} \PY{n+nf}{show\PYZus{}fourier}\PY{p}{(}\PY{n}{folder} \PY{o}{=} \PY{l+s+s2}{\PYZdq{}}\PY{l+s+s2}{/content/drive/MyDrive/deepfake\PYZus{}dataset/fourier}\PY{l+s+s2}{\PYZdq{}}\PY{p}{,} \PY{n}{total\PYZus{}filename}\PY{o}{=}\PY{l+s+s2}{\PYZdq{}}\PY{l+s+s2}{total.png}\PY{l+s+s2}{\PYZdq{}}\PY{p}{)}\PY{p}{:}
    \PY{n}{total\PYZus{}filepath} \PY{o}{=} \PY{n}{os}\PY{o}{.}\PY{n}{path}\PY{o}{.}\PY{n}{join}\PY{p}{(}\PY{n}{folder}\PY{p}{,} \PY{n}{total\PYZus{}filename}\PY{p}{)}

    \PY{k}{if} \PY{n}{os}\PY{o}{.}\PY{n}{path}\PY{o}{.}\PY{n}{exists}\PY{p}{(}\PY{n}{total\PYZus{}filepath}\PY{p}{)}\PY{p}{:}
        \PY{k}{return} \PY{n}{ipydis}\PY{p}{(}\PY{n}{filename}\PY{o}{=}\PY{n}{total\PYZus{}filepath}\PY{p}{)}

    \PY{n}{fourier\PYZus{}images} \PY{o}{=} \PY{p}{[}
        \PY{n}{os}\PY{o}{.}\PY{n}{path}\PY{o}{.}\PY{n}{join}\PY{p}{(}\PY{n}{input\PYZus{}folder}\PY{p}{,} \PY{n}{file}\PY{p}{)} \PY{k}{for} \PY{n}{file} \PY{o+ow}{in} \PY{n}{os}\PY{o}{.}\PY{n}{listdir}\PY{p}{(}\PY{n}{input\PYZus{}folder}\PY{p}{)}
        \PY{k}{if} \PY{n}{file}\PY{o}{.}\PY{n}{endswith}\PY{p}{(}\PY{l+s+s2}{\PYZdq{}}\PY{l+s+s2}{.png}\PY{l+s+s2}{\PYZdq{}}\PY{p}{)}
    \PY{p}{]}

    \PY{n}{cols} \PY{o}{=} \PY{l+m+mi}{3}
    \PY{n}{rows} \PY{o}{=} \PY{p}{(}\PY{n+nb}{len}\PY{p}{(}\PY{n}{fourier\PYZus{}images}\PY{p}{)} \PY{o}{+} \PY{n}{cols} \PY{o}{\PYZhy{}} \PY{l+m+mi}{1}\PY{p}{)} \PY{o}{/}\PY{o}{/} \PY{n}{cols}

    \PY{n}{fig}\PY{p}{,} \PY{n}{axes} \PY{o}{=} \PY{n}{plt}\PY{o}{.}\PY{n}{subplots}\PY{p}{(}\PY{n}{rows}\PY{p}{,} \PY{n}{cols}\PY{p}{,} \PY{n}{figsize}\PY{o}{=}\PY{p}{(}\PY{l+m+mi}{15}\PY{p}{,} \PY{l+m+mi}{5} \PY{o}{*} \PY{n}{rows}\PY{p}{)}\PY{p}{)}
    \PY{n}{axes} \PY{o}{=} \PY{n}{axes}\PY{o}{.}\PY{n}{ravel}\PY{p}{(}\PY{p}{)}

    \PY{k}{for} \PY{n}{i}\PY{p}{,} \PY{n}{img\PYZus{}path} \PY{o+ow}{in} \PY{n+nb}{enumerate}\PY{p}{(}\PY{n}{fourier\PYZus{}images}\PY{p}{)}\PY{p}{:}
        \PY{n}{img\PYZus{}data} \PY{o}{=} \PY{n}{imread}\PY{p}{(}\PY{n}{img\PYZus{}path}\PY{p}{)}
        \PY{n}{axes}\PY{p}{[}\PY{n}{i}\PY{p}{]}\PY{o}{.}\PY{n}{imshow}\PY{p}{(}\PY{n}{img\PYZus{}data}\PY{p}{,} \PY{n}{cmap}\PY{o}{=}\PY{l+s+s2}{\PYZdq{}}\PY{l+s+s2}{viridis}\PY{l+s+s2}{\PYZdq{}}\PY{p}{)}
        \PY{n}{axes}\PY{p}{[}\PY{n}{i}\PY{p}{]}\PY{o}{.}\PY{n}{set\PYZus{}title}\PY{p}{(}\PY{n}{os}\PY{o}{.}\PY{n}{path}\PY{o}{.}\PY{n}{basename}\PY{p}{(}\PY{n}{img\PYZus{}path}\PY{p}{)}\PY{p}{,} \PY{n}{fontsize}\PY{o}{=}\PY{l+m+mi}{10}\PY{p}{)}
        \PY{n}{axes}\PY{p}{[}\PY{n}{i}\PY{p}{]}\PY{o}{.}\PY{n}{axis}\PY{p}{(}\PY{l+s+s2}{\PYZdq{}}\PY{l+s+s2}{off}\PY{l+s+s2}{\PYZdq{}}\PY{p}{)}

    \PY{k}{for} \PY{n}{j} \PY{o+ow}{in} \PY{n+nb}{range}\PY{p}{(}\PY{n+nb}{len}\PY{p}{(}\PY{n}{fourier\PYZus{}images}\PY{p}{)}\PY{p}{,} \PY{n+nb}{len}\PY{p}{(}\PY{n}{axes}\PY{p}{)}\PY{p}{)}\PY{p}{:}
        \PY{n}{axes}\PY{p}{[}\PY{n}{j}\PY{p}{]}\PY{o}{.}\PY{n}{axis}\PY{p}{(}\PY{l+s+s2}{\PYZdq{}}\PY{l+s+s2}{off}\PY{l+s+s2}{\PYZdq{}}\PY{p}{)}

    \PY{n}{plt}\PY{o}{.}\PY{n}{tight\PYZus{}layout}\PY{p}{(}\PY{p}{)}
    \PY{n}{plt}\PY{o}{.}\PY{n}{savefig}\PY{p}{(}\PY{n}{total\PYZus{}filepath}\PY{p}{,} \PY{n}{dpi}\PY{o}{=}\PY{l+m+mi}{300}\PY{p}{,} \PY{n}{bbox\PYZus{}inches}\PY{o}{=}\PY{l+s+s1}{\PYZsq{}}\PY{l+s+s1}{tight}\PY{l+s+s1}{\PYZsq{}}\PY{p}{)}
    \PY{n}{plt}\PY{o}{.}\PY{n}{close}\PY{p}{(}\PY{p}{)}
    \PY{n+nb}{print}\PY{p}{(}\PY{l+s+sa}{f}\PY{l+s+s2}{\PYZdq{}}\PY{l+s+s2}{Combined Fourier image saved: }\PY{l+s+si}{\PYZob{}}\PY{n}{total\PYZus{}filepath}\PY{l+s+si}{\PYZcb{}}\PY{l+s+s2}{\PYZdq{}}\PY{p}{)}
\end{Verbatim}
\end{tcolorbox}

    \subsection{To Train}\label{to-train}

Here we have the four functions used to train each model: 1. the
function to get the criterion and optimizer; 2. two functions that
handle training the model within a single epoch and evaluating it; 3. A
total function that loops over the epochs using the two previous
functions.

    \begin{tcolorbox}[breakable, size=fbox, boxrule=1pt, pad at break*=1mm,colback=cellbackground, colframe=cellborder]
\prompt{In}{incolor}{5}{\boxspacing}
\begin{Verbatim}[commandchars=\\\{\}]
\PY{c+c1}{\PYZsh{}FUNCTION FOR CRITERION \PYZam{} OPTIMIZER (ADAM)}
\PY{k}{def} \PY{n+nf}{setup\PYZus{}training}\PY{p}{(}\PY{n}{model}\PY{p}{,} \PY{n}{learning\PYZus{}rate}\PY{o}{=}\PY{l+m+mf}{1e\PYZhy{}4}\PY{p}{)}\PY{p}{:}
    \PY{n}{criterion} \PY{o}{=} \PY{n}{nn}\PY{o}{.}\PY{n}{CrossEntropyLoss}\PY{p}{(}\PY{p}{)}
    \PY{n}{optimizer} \PY{o}{=} \PY{n}{optim}\PY{o}{.}\PY{n}{Adam}\PY{p}{(}\PY{n}{model}\PY{o}{.}\PY{n}{parameters}\PY{p}{(}\PY{p}{)}\PY{p}{,} \PY{n}{lr}\PY{o}{=}\PY{n}{learning\PYZus{}rate}\PY{p}{)}
    \PY{n}{scheduler} \PY{o}{=} \PY{n}{torch}\PY{o}{.}\PY{n}{optim}\PY{o}{.}\PY{n}{lr\PYZus{}scheduler}\PY{o}{.}\PY{n}{ReduceLROnPlateau}\PY{p}{(}\PY{n}{optimizer}\PY{p}{,} \PY{n}{mode}\PY{o}{=}\PY{l+s+s1}{\PYZsq{}}\PY{l+s+s1}{min}\PY{l+s+s1}{\PYZsq{}}\PY{p}{,} \PY{n}{patience}\PY{o}{=}\PY{l+m+mi}{2}\PY{p}{,} \PY{n}{factor}\PY{o}{=}\PY{l+m+mf}{0.5}\PY{p}{,} \PY{n}{verbose}\PY{o}{=}\PY{k+kc}{True}\PY{p}{)}

    \PY{k}{return} \PY{n}{criterion}\PY{p}{,} \PY{n}{optimizer}\PY{p}{,} \PY{n}{scheduler}
\end{Verbatim}
\end{tcolorbox}

    \begin{tcolorbox}[breakable, size=fbox, boxrule=1pt, pad at break*=1mm,colback=cellbackground, colframe=cellborder]
\prompt{In}{incolor}{6}{\boxspacing}
\begin{Verbatim}[commandchars=\\\{\}]
\PY{c+c1}{\PYZsh{}TRAINING FUNCTION}
\PY{k}{def} \PY{n+nf}{train\PYZus{}one\PYZus{}epoch}\PY{p}{(}\PY{n}{model}\PY{p}{,} \PY{n}{train\PYZus{}loader}\PY{p}{,} \PY{n}{criterion}\PY{p}{,} \PY{n}{optimizer}\PY{p}{,} \PY{n}{device}\PY{p}{,} \PY{n}{accumulation\PYZus{}steps}\PY{o}{=}\PY{l+m+mi}{4}\PY{p}{)}\PY{p}{:}
    \PY{n}{model}\PY{o}{.}\PY{n}{train}\PY{p}{(}\PY{p}{)}  \PY{c+c1}{\PYZsh{}training mode}
    \PY{n}{train\PYZus{}loss}\PY{p}{,} \PY{n}{train\PYZus{}correct} \PY{o}{=} \PY{l+m+mf}{0.0}\PY{p}{,} \PY{l+m+mi}{0}

    \PY{k}{for} \PY{n}{i}\PY{p}{,} \PY{p}{(}\PY{n}{images}\PY{p}{,} \PY{n}{labels}\PY{p}{)} \PY{o+ow}{in} \PY{n+nb}{enumerate}\PY{p}{(}\PY{n}{train\PYZus{}loader}\PY{p}{)}\PY{p}{:}
        \PY{c+c1}{\PYZsh{} moves the data to the GPU or CPU}
        \PY{n}{images}\PY{p}{,} \PY{n}{labels} \PY{o}{=} \PY{n}{images}\PY{o}{.}\PY{n}{to}\PY{p}{(}\PY{n}{device}\PY{p}{)}\PY{p}{,} \PY{n}{labels}\PY{o}{.}\PY{n}{to}\PY{p}{(}\PY{n}{device}\PY{p}{)}

        \PY{c+c1}{\PYZsh{} forward pass}
        \PY{n}{outputs} \PY{o}{=} \PY{n}{model}\PY{p}{(}\PY{n}{images}\PY{p}{)}
        \PY{n}{outputs} \PY{o}{=} \PY{n}{outputs}\PY{o}{.}\PY{n}{logits} \PY{k}{if} \PY{n+nb}{hasattr}\PY{p}{(}\PY{n}{outputs}\PY{p}{,} \PY{l+s+s1}{\PYZsq{}}\PY{l+s+s1}{logits}\PY{l+s+s1}{\PYZsq{}}\PY{p}{)} \PY{k}{else} \PY{n}{outputs}

        \PY{n}{loss} \PY{o}{=} \PY{n}{criterion}\PY{p}{(}\PY{n}{outputs}\PY{p}{,} \PY{n}{labels}\PY{p}{)}

        \PY{c+c1}{\PYZsh{} accumulated gradient}
        \PY{n}{loss} \PY{o}{=} \PY{n}{loss} \PY{o}{/} \PY{n}{accumulation\PYZus{}steps}
        \PY{n}{loss}\PY{o}{.}\PY{n}{backward}\PY{p}{(}\PY{p}{)}

        \PY{c+c1}{\PYZsh{} Update the weights every \PYZsq{}accumulation\PYZus{}steps\PYZsq{} steps}
        \PY{k}{if} \PY{p}{(}\PY{n}{i} \PY{o}{+} \PY{l+m+mi}{1}\PY{p}{)} \PY{o}{\PYZpc{}} \PY{n}{accumulation\PYZus{}steps} \PY{o}{==} \PY{l+m+mi}{0}\PY{p}{:}
            \PY{n}{optimizer}\PY{o}{.}\PY{n}{step}\PY{p}{(}\PY{p}{)}
            \PY{n}{optimizer}\PY{o}{.}\PY{n}{zero\PYZus{}grad}\PY{p}{(}\PY{p}{)}

        \PY{c+c1}{\PYZsh{} Tracking loss and accuracy}
        \PY{n}{train\PYZus{}loss} \PY{o}{+}\PY{o}{=} \PY{n}{loss}\PY{o}{.}\PY{n}{item}\PY{p}{(}\PY{p}{)} \PY{o}{*} \PY{n}{images}\PY{o}{.}\PY{n}{size}\PY{p}{(}\PY{l+m+mi}{0}\PY{p}{)}
        \PY{n}{\PYZus{}}\PY{p}{,} \PY{n}{preds} \PY{o}{=} \PY{n}{torch}\PY{o}{.}\PY{n}{max}\PY{p}{(}\PY{n}{outputs}\PY{p}{,} \PY{l+m+mi}{1}\PY{p}{)}
        \PY{n}{train\PYZus{}correct} \PY{o}{+}\PY{o}{=} \PY{n}{torch}\PY{o}{.}\PY{n}{sum}\PY{p}{(}\PY{n}{preds} \PY{o}{==} \PY{n}{labels}\PY{o}{.}\PY{n}{data}\PY{p}{)}\PY{o}{.}\PY{n}{item}\PY{p}{(}\PY{p}{)}

    \PY{c+c1}{\PYZsh{} Loss calculation and average accuracy}
    \PY{n}{epoch\PYZus{}loss} \PY{o}{=} \PY{n}{train\PYZus{}loss} \PY{o}{/} \PY{n+nb}{len}\PY{p}{(}\PY{n}{train\PYZus{}loader}\PY{o}{.}\PY{n}{dataset}\PY{p}{)}
    \PY{n}{epoch\PYZus{}acc} \PY{o}{=} \PY{n}{train\PYZus{}correct} \PY{o}{/} \PY{n+nb}{len}\PY{p}{(}\PY{n}{train\PYZus{}loader}\PY{o}{.}\PY{n}{dataset}\PY{p}{)}

    \PY{k}{return} \PY{n}{epoch\PYZus{}loss}\PY{p}{,} \PY{n}{epoch\PYZus{}acc}
\end{Verbatim}
\end{tcolorbox}

    \begin{tcolorbox}[breakable, size=fbox, boxrule=1pt, pad at break*=1mm,colback=cellbackground, colframe=cellborder]
\prompt{In}{incolor}{7}{\boxspacing}
\begin{Verbatim}[commandchars=\\\{\}]
\PY{c+c1}{\PYZsh{}VALIDATION FUNCTION}
\PY{k}{def} \PY{n+nf}{validate\PYZus{}one\PYZus{}epoch}\PY{p}{(}\PY{n}{model}\PY{p}{,} \PY{n}{val\PYZus{}loader}\PY{p}{,} \PY{n}{criterion}\PY{p}{,} \PY{n}{device}\PY{p}{)}\PY{p}{:}
    \PY{n}{model}\PY{o}{.}\PY{n}{eval}\PY{p}{(}\PY{p}{)}  \PY{c+c1}{\PYZsh{} Validation mode}
    \PY{n}{val\PYZus{}loss} \PY{o}{=} \PY{l+m+mf}{0.0}
    \PY{n}{val\PYZus{}correct} \PY{o}{=} \PY{l+m+mi}{0}
    \PY{n}{val\PYZus{}total} \PY{o}{=} \PY{l+m+mi}{0}
    \PY{n}{all\PYZus{}labels} \PY{o}{=} \PY{p}{[}\PY{p}{]}
    \PY{n}{all\PYZus{}preds} \PY{o}{=} \PY{p}{[}\PY{p}{]}

    \PY{k}{with} \PY{n}{torch}\PY{o}{.}\PY{n}{no\PYZus{}grad}\PY{p}{(}\PY{p}{)}\PY{p}{:}  \PY{c+c1}{\PYZsh{} Disable gradient calculation}
        \PY{k}{for} \PY{n}{images}\PY{p}{,} \PY{n}{labels} \PY{o+ow}{in} \PY{n}{val\PYZus{}loader}\PY{p}{:}
            \PY{c+c1}{\PYZsh{} Moves data to the GPU or CPU}
            \PY{n}{images}\PY{p}{,} \PY{n}{labels} \PY{o}{=} \PY{n}{images}\PY{o}{.}\PY{n}{to}\PY{p}{(}\PY{n}{device}\PY{p}{)}\PY{p}{,} \PY{n}{labels}\PY{o}{.}\PY{n}{to}\PY{p}{(}\PY{n}{device}\PY{p}{)}

            \PY{c+c1}{\PYZsh{} Moving forward}
            \PY{n}{outputs} \PY{o}{=} \PY{n}{model}\PY{p}{(}\PY{n}{images}\PY{p}{)}
            \PY{n}{outputs} \PY{o}{=} \PY{n}{outputs}\PY{o}{.}\PY{n}{logits} \PY{k}{if} \PY{n+nb}{hasattr}\PY{p}{(}\PY{n}{outputs}\PY{p}{,} \PY{l+s+s1}{\PYZsq{}}\PY{l+s+s1}{logits}\PY{l+s+s1}{\PYZsq{}}\PY{p}{)} \PY{k}{else} \PY{n}{outputs}

            \PY{n}{loss} \PY{o}{=} \PY{n}{criterion}\PY{p}{(}\PY{n}{outputs}\PY{p}{,} \PY{n}{labels}\PY{p}{)}

            \PY{c+c1}{\PYZsh{} Tracking loss and accuracy}
            \PY{n}{val\PYZus{}loss} \PY{o}{+}\PY{o}{=} \PY{n}{loss}\PY{o}{.}\PY{n}{item}\PY{p}{(}\PY{p}{)} \PY{o}{*} \PY{n}{images}\PY{o}{.}\PY{n}{size}\PY{p}{(}\PY{l+m+mi}{0}\PY{p}{)}
            \PY{n}{\PYZus{}}\PY{p}{,} \PY{n}{preds} \PY{o}{=} \PY{n}{torch}\PY{o}{.}\PY{n}{max}\PY{p}{(}\PY{n}{outputs}\PY{p}{,} \PY{l+m+mi}{1}\PY{p}{)}
            \PY{n}{val\PYZus{}correct} \PY{o}{+}\PY{o}{=} \PY{n}{torch}\PY{o}{.}\PY{n}{sum}\PY{p}{(}\PY{n}{preds} \PY{o}{==} \PY{n}{labels}\PY{o}{.}\PY{n}{data}\PY{p}{)}\PY{o}{.}\PY{n}{item}\PY{p}{(}\PY{p}{)}
            \PY{n}{val\PYZus{}total} \PY{o}{+}\PY{o}{=} \PY{n}{labels}\PY{o}{.}\PY{n}{size}\PY{p}{(}\PY{l+m+mi}{0}\PY{p}{)}

            \PY{n}{all\PYZus{}labels}\PY{o}{.}\PY{n}{extend}\PY{p}{(}\PY{n}{labels}\PY{o}{.}\PY{n}{cpu}\PY{p}{(}\PY{p}{)}\PY{o}{.}\PY{n}{numpy}\PY{p}{(}\PY{p}{)}\PY{p}{)}
            \PY{n}{all\PYZus{}preds}\PY{o}{.}\PY{n}{extend}\PY{p}{(}\PY{n}{preds}\PY{o}{.}\PY{n}{cpu}\PY{p}{(}\PY{p}{)}\PY{o}{.}\PY{n}{numpy}\PY{p}{(}\PY{p}{)}\PY{p}{)}

    \PY{c+c1}{\PYZsh{} Calculates average loss and accuracy}
    \PY{n}{val\PYZus{}loss} \PY{o}{=} \PY{n}{val\PYZus{}loss} \PY{o}{/} \PY{n}{val\PYZus{}total}
    \PY{n}{val\PYZus{}accuracy} \PY{o}{=} \PY{n}{val\PYZus{}correct} \PY{o}{/} \PY{n}{val\PYZus{}total}

    \PY{k}{return} \PY{n}{val\PYZus{}loss}\PY{p}{,} \PY{n}{val\PYZus{}accuracy}\PY{p}{,} \PY{n}{all\PYZus{}labels}\PY{p}{,} \PY{n}{all\PYZus{}preds}
\end{Verbatim}
\end{tcolorbox}

    \begin{tcolorbox}[breakable, size=fbox, boxrule=1pt, pad at break*=1mm,colback=cellbackground, colframe=cellborder]
\prompt{In}{incolor}{8}{\boxspacing}
\begin{Verbatim}[commandchars=\\\{\}]
\PY{k}{def} \PY{n+nf}{train\PYZus{}model}\PY{p}{(}\PY{n}{model}\PY{p}{,} \PY{n}{model\PYZus{}name}\PY{p}{,} \PY{n}{criterion}\PY{p}{,} \PY{n}{optimizer}\PY{p}{,} \PY{n}{train\PYZus{}loader}\PY{p}{,} \PY{n}{val\PYZus{}loader}\PY{p}{,} \PY{n}{num\PYZus{}epochs}\PY{o}{=}\PY{l+m+mi}{20}\PY{p}{,} \PY{n}{accumulation\PYZus{}steps}\PY{o}{=}\PY{l+m+mi}{4}\PY{p}{,} \PY{n}{patience}\PY{o}{=}\PY{l+m+mi}{3}\PY{p}{)}\PY{p}{:}

    \PY{n}{device} \PY{o}{=} \PY{n}{torch}\PY{o}{.}\PY{n}{device}\PY{p}{(}\PY{l+s+s1}{\PYZsq{}}\PY{l+s+s1}{cuda}\PY{l+s+s1}{\PYZsq{}} \PY{k}{if} \PY{n}{torch}\PY{o}{.}\PY{n}{cuda}\PY{o}{.}\PY{n}{is\PYZus{}available}\PY{p}{(}\PY{p}{)} \PY{k}{else} \PY{l+s+s1}{\PYZsq{}}\PY{l+s+s1}{cpu}\PY{l+s+s1}{\PYZsq{}}\PY{p}{)}
    \PY{n}{model} \PY{o}{=} \PY{n}{model}\PY{o}{.}\PY{n}{to}\PY{p}{(}\PY{n}{device}\PY{p}{)}  \PY{c+c1}{\PYZsh{} Moves the model to the GPU or CPU}

    \PY{c+c1}{\PYZsh{} Directory for saving model\PYZhy{}related files}
    \PY{n}{model\PYZus{}dir} \PY{o}{=} \PY{l+s+sa}{f}\PY{l+s+s1}{\PYZsq{}}\PY{l+s+s1}{/content/drive/MyDrive/deepfake\PYZus{}dataset/models/}\PY{l+s+si}{\PYZob{}}\PY{n}{model\PYZus{}name}\PY{l+s+si}{\PYZcb{}}\PY{l+s+s1}{\PYZsq{}}

    \PY{c+c1}{\PYZsh{} Check if the model directory already exists}
    \PY{k}{if} \PY{n}{os}\PY{o}{.}\PY{n}{path}\PY{o}{.}\PY{n}{exists}\PY{p}{(}\PY{n}{model\PYZus{}dir}\PY{p}{)}\PY{p}{:}
        \PY{n+nb}{print}\PY{p}{(}\PY{l+s+sa}{f}\PY{l+s+s2}{\PYZdq{}}\PY{l+s+se}{\PYZbs{}n}\PY{l+s+s2}{Model }\PY{l+s+s2}{\PYZsq{}}\PY{l+s+si}{\PYZob{}}\PY{n}{model\PYZus{}name}\PY{l+s+si}{\PYZcb{}}\PY{l+s+s2}{\PYZsq{}}\PY{l+s+s2}{ already exists in }\PY{l+s+si}{\PYZob{}}\PY{n}{model\PYZus{}dir}\PY{l+s+si}{\PYZcb{}}\PY{l+s+s2}{. Training skipped.}\PY{l+s+s2}{\PYZdq{}}\PY{p}{)}
        \PY{k}{return} \PY{k+kc}{None}  \PY{c+c1}{\PYZsh{} Exit the function without training}

    \PY{c+c1}{\PYZsh{} Create the directory for the model if it does not exist}
    \PY{n}{os}\PY{o}{.}\PY{n}{makedirs}\PY{p}{(}\PY{n}{model\PYZus{}dir}\PY{p}{)}

    \PY{c+c1}{\PYZsh{} Structure of results for the model}
    \PY{n}{model\PYZus{}results} \PY{o}{=} \PY{p}{\PYZob{}}
        \PY{l+s+s2}{\PYZdq{}}\PY{l+s+s2}{epoch\PYZus{}results}\PY{l+s+s2}{\PYZdq{}}\PY{p}{:} \PY{p}{[}\PY{p}{]}\PY{p}{,}  \PY{c+c1}{\PYZsh{} Results for each epoch}
        \PY{l+s+s2}{\PYZdq{}}\PY{l+s+s2}{confusion\PYZus{}matrix}\PY{l+s+s2}{\PYZdq{}}\PY{p}{:} \PY{k+kc}{None}  \PY{c+c1}{\PYZsh{} Confusion matrix}
    \PY{p}{\PYZcb{}}

    \PY{n}{best\PYZus{}val\PYZus{}loss} \PY{o}{=} \PY{n+nb}{float}\PY{p}{(}\PY{l+s+s1}{\PYZsq{}}\PY{l+s+s1}{inf}\PY{l+s+s1}{\PYZsq{}}\PY{p}{)}  \PY{c+c1}{\PYZsh{} Initialise best validation loss as infinite}
    \PY{n}{epochs\PYZus{}since\PYZus{}improvement} \PY{o}{=} \PY{l+m+mi}{0}  \PY{c+c1}{\PYZsh{} Count the number of epochs without improvement}

    \PY{c+c1}{\PYZsh{} Initialise global progress bar}
    \PY{n}{progress\PYZus{}bar} \PY{o}{=} \PY{n}{tqdm}\PY{p}{(}
        \PY{n}{total}\PY{o}{=}\PY{n}{num\PYZus{}epochs}\PY{p}{,}
        \PY{n}{desc}\PY{o}{=}\PY{l+s+s2}{\PYZdq{}}\PY{l+s+s2}{Training Progress}\PY{l+s+s2}{\PYZdq{}}\PY{p}{,}
        \PY{n}{ncols}\PY{o}{=}\PY{l+m+mi}{700}\PY{p}{,}
        \PY{n}{leave}\PY{o}{=}\PY{k+kc}{True}\PY{p}{,}
        \PY{n}{bar\PYZus{}format}\PY{o}{=}\PY{l+s+s2}{\PYZdq{}}\PY{l+s+si}{\PYZob{}desc\PYZcb{}}\PY{l+s+s2}{ }\PY{l+s+si}{\PYZob{}percentage:3.0f\PYZcb{}}\PY{l+s+s2}{\PYZpc{}}\PY{l+s+s2}{|}\PY{l+s+si}{\PYZob{}bar\PYZcb{}}\PY{l+s+s2}{| }\PY{l+s+si}{\PYZob{}n\PYZus{}fmt\PYZcb{}}\PY{l+s+s2}{/}\PY{l+s+si}{\PYZob{}total\PYZus{}fmt\PYZcb{}}\PY{l+s+s2}{ epochs}\PY{l+s+s2}{\PYZdq{}}
    \PY{p}{)}

    \PY{k}{for} \PY{n}{epoch} \PY{o+ow}{in} \PY{n+nb}{range}\PY{p}{(}\PY{n}{num\PYZus{}epochs}\PY{p}{)}\PY{p}{:}

        \PY{c+c1}{\PYZsh{} Training phase}
        \PY{n}{train\PYZus{}loss}\PY{p}{,} \PY{n}{train\PYZus{}acc} \PY{o}{=} \PY{n}{train\PYZus{}one\PYZus{}epoch}\PY{p}{(}\PY{n}{model}\PY{p}{,} \PY{n}{train\PYZus{}loader}\PY{p}{,} \PY{n}{criterion}\PY{p}{,} \PY{n}{optimizer}\PY{p}{,} \PY{n}{device}\PY{p}{,} \PY{n}{accumulation\PYZus{}steps}\PY{p}{)}

        \PY{c+c1}{\PYZsh{} Validation phase}
        \PY{n}{val\PYZus{}loss}\PY{p}{,} \PY{n}{val\PYZus{}acc}\PY{p}{,} \PY{n}{all\PYZus{}labels}\PY{p}{,} \PY{n}{all\PYZus{}preds} \PY{o}{=} \PY{n}{validate\PYZus{}one\PYZus{}epoch}\PY{p}{(}\PY{n}{model}\PY{p}{,} \PY{n}{val\PYZus{}loader}\PY{p}{,} \PY{n}{criterion}\PY{p}{,} \PY{n}{device}\PY{p}{)}

        \PY{c+c1}{\PYZsh{} Confusion matrix calculation and metrics and add it to the list}
        \PY{n}{precision} \PY{o}{=} \PY{n}{precision\PYZus{}score}\PY{p}{(}\PY{n}{all\PYZus{}labels}\PY{p}{,} \PY{n}{all\PYZus{}preds}\PY{p}{,} \PY{n}{average}\PY{o}{=}\PY{l+s+s1}{\PYZsq{}}\PY{l+s+s1}{binary}\PY{l+s+s1}{\PYZsq{}}\PY{p}{)}
        \PY{n}{recall} \PY{o}{=} \PY{n}{recall\PYZus{}score}\PY{p}{(}\PY{n}{all\PYZus{}labels}\PY{p}{,} \PY{n}{all\PYZus{}preds}\PY{p}{,} \PY{n}{average}\PY{o}{=}\PY{l+s+s1}{\PYZsq{}}\PY{l+s+s1}{binary}\PY{l+s+s1}{\PYZsq{}}\PY{p}{)}
        \PY{n}{f1} \PY{o}{=} \PY{n}{f1\PYZus{}score}\PY{p}{(}\PY{n}{all\PYZus{}labels}\PY{p}{,} \PY{n}{all\PYZus{}preds}\PY{p}{,} \PY{n}{average}\PY{o}{=}\PY{l+s+s1}{\PYZsq{}}\PY{l+s+s1}{binary}\PY{l+s+s1}{\PYZsq{}}\PY{p}{)}
        \PY{n}{cm} \PY{o}{=} \PY{n}{confusion\PYZus{}matrix}\PY{p}{(}\PY{n}{all\PYZus{}labels}\PY{p}{,} \PY{n}{all\PYZus{}preds}\PY{p}{)}

        \PY{n}{model\PYZus{}results}\PY{p}{[}\PY{l+s+s2}{\PYZdq{}}\PY{l+s+s2}{epoch\PYZus{}results}\PY{l+s+s2}{\PYZdq{}}\PY{p}{]}\PY{o}{.}\PY{n}{append}\PY{p}{(}\PY{p}{\PYZob{}}
            \PY{l+s+s1}{\PYZsq{}}\PY{l+s+s1}{Epoch}\PY{l+s+s1}{\PYZsq{}}\PY{p}{:} \PY{n}{epoch} \PY{o}{+} \PY{l+m+mi}{1}\PY{p}{,}
            \PY{l+s+s1}{\PYZsq{}}\PY{l+s+s1}{Train Loss}\PY{l+s+s1}{\PYZsq{}}\PY{p}{:} \PY{n+nb}{round}\PY{p}{(}\PY{n}{train\PYZus{}loss}\PY{p}{,} \PY{l+m+mi}{4}\PY{p}{)}\PY{p}{,}
            \PY{l+s+s1}{\PYZsq{}}\PY{l+s+s1}{Train Accuracy}\PY{l+s+s1}{\PYZsq{}}\PY{p}{:} \PY{n+nb}{round}\PY{p}{(}\PY{n}{train\PYZus{}acc}\PY{p}{,} \PY{l+m+mi}{4}\PY{p}{)}\PY{p}{,}
            \PY{l+s+s1}{\PYZsq{}}\PY{l+s+s1}{Validation Loss}\PY{l+s+s1}{\PYZsq{}}\PY{p}{:} \PY{n+nb}{round}\PY{p}{(}\PY{n}{val\PYZus{}loss}\PY{p}{,} \PY{l+m+mi}{4}\PY{p}{)}\PY{p}{,}
            \PY{l+s+s1}{\PYZsq{}}\PY{l+s+s1}{Validation Accuracy}\PY{l+s+s1}{\PYZsq{}}\PY{p}{:} \PY{n+nb}{round}\PY{p}{(}\PY{n}{val\PYZus{}acc}\PY{p}{,} \PY{l+m+mi}{4}\PY{p}{)}\PY{p}{,}
            \PY{l+s+s1}{\PYZsq{}}\PY{l+s+s1}{Precision}\PY{l+s+s1}{\PYZsq{}}\PY{p}{:} \PY{n+nb}{round}\PY{p}{(}\PY{n}{precision}\PY{p}{,} \PY{l+m+mi}{4}\PY{p}{)}\PY{p}{,}
            \PY{l+s+s1}{\PYZsq{}}\PY{l+s+s1}{Recall}\PY{l+s+s1}{\PYZsq{}}\PY{p}{:} \PY{n+nb}{round}\PY{p}{(}\PY{n}{recall}\PY{p}{,} \PY{l+m+mi}{4}\PY{p}{)}\PY{p}{,}
            \PY{l+s+s1}{\PYZsq{}}\PY{l+s+s1}{F1 Score}\PY{l+s+s1}{\PYZsq{}}\PY{p}{:} \PY{n+nb}{round}\PY{p}{(}\PY{n}{f1}\PY{p}{,} \PY{l+m+mi}{4}\PY{p}{)}
        \PY{p}{\PYZcb{}}\PY{p}{)}
        \PY{n}{model\PYZus{}results}\PY{p}{[}\PY{l+s+s2}{\PYZdq{}}\PY{l+s+s2}{confusion\PYZus{}matrix}\PY{l+s+s2}{\PYZdq{}}\PY{p}{]} \PY{o}{=} \PY{n}{cm}

        \PY{n}{scheduler}\PY{o}{.}\PY{n}{step}\PY{p}{(}\PY{n}{val\PYZus{}loss}\PY{p}{)}

        \PY{c+c1}{\PYZsh{} Check if the validation loss has improved}
        \PY{k}{if} \PY{n}{val\PYZus{}loss} \PY{o}{\PYZlt{}} \PY{n}{best\PYZus{}val\PYZus{}loss}\PY{p}{:}
            \PY{n}{best\PYZus{}val\PYZus{}loss} \PY{o}{=} \PY{n}{val\PYZus{}loss}  \PY{c+c1}{\PYZsh{} Update best validation loss}
            \PY{n}{epochs\PYZus{}since\PYZus{}improvement} \PY{o}{=} \PY{l+m+mi}{0}  \PY{c+c1}{\PYZsh{} Resets the epoch counter without improvement}
        \PY{k}{else}\PY{p}{:}
            \PY{n}{epochs\PYZus{}since\PYZus{}improvement} \PY{o}{+}\PY{o}{=} \PY{l+m+mi}{1}

        \PY{c+c1}{\PYZsh{} Update progress bar}
        \PY{n}{progress\PYZus{}bar}\PY{o}{.}\PY{n}{update}\PY{p}{(}\PY{l+m+mi}{1}\PY{p}{)}

        \PY{c+c1}{\PYZsh{} If there is no improvement for \PYZsq{}patience\PYZsq{} epochs, stop training}
        \PY{k}{if} \PY{n}{epochs\PYZus{}since\PYZus{}improvement} \PY{o}{\PYZgt{}}\PY{o}{=} \PY{n}{patience}\PY{p}{:}
            \PY{n+nb}{print}\PY{p}{(}\PY{l+s+s2}{\PYZdq{}}\PY{l+s+se}{\PYZbs{}n}\PY{l+s+s2}{Early stopping: No improvement in validation loss.}\PY{l+s+s2}{\PYZdq{}}\PY{p}{)}
            \PY{k}{break}

        \PY{c+c1}{\PYZsh{} Release the GPU memory at the end of each era}
        \PY{n}{torch}\PY{o}{.}\PY{n}{cuda}\PY{o}{.}\PY{n}{empty\PYZus{}cache}\PY{p}{(}\PY{p}{)}
    \PY{n}{progress\PYZus{}bar}\PY{o}{.}\PY{n}{close}\PY{p}{(}\PY{p}{)}

    \PY{c+c1}{\PYZsh{} After completing the training, move the model back to the CPU}
    \PY{n}{model} \PY{o}{=} \PY{n}{model}\PY{o}{.}\PY{n}{to}\PY{p}{(}\PY{l+s+s1}{\PYZsq{}}\PY{l+s+s1}{cpu}\PY{l+s+s1}{\PYZsq{}}\PY{p}{)}

    \PY{c+c1}{\PYZsh{} Saves the results in a CSV file in the model folder}
    \PY{n}{results\PYZus{}df} \PY{o}{=} \PY{n}{pd}\PY{o}{.}\PY{n}{DataFrame}\PY{p}{(}\PY{n}{model\PYZus{}results}\PY{p}{[}\PY{l+s+s2}{\PYZdq{}}\PY{l+s+s2}{epoch\PYZus{}results}\PY{l+s+s2}{\PYZdq{}}\PY{p}{]}\PY{p}{)}
    \PY{n}{results\PYZus{}df}\PY{o}{.}\PY{n}{to\PYZus{}csv}\PY{p}{(}\PY{l+s+sa}{f}\PY{l+s+s1}{\PYZsq{}}\PY{l+s+si}{\PYZob{}}\PY{n}{model\PYZus{}dir}\PY{l+s+si}{\PYZcb{}}\PY{l+s+s1}{/}\PY{l+s+si}{\PYZob{}}\PY{n}{model\PYZus{}name}\PY{l+s+si}{\PYZcb{}}\PY{l+s+s1}{\PYZus{}results.csv}\PY{l+s+s1}{\PYZsq{}}\PY{p}{,} \PY{n}{index}\PY{o}{=}\PY{k+kc}{False}\PY{p}{)}
    \PY{n+nb}{print}\PY{p}{(}\PY{l+s+sa}{f}\PY{l+s+s2}{\PYZdq{}}\PY{l+s+se}{\PYZbs{}n}\PY{l+s+s2}{Training results for }\PY{l+s+si}{\PYZob{}}\PY{n}{model\PYZus{}name}\PY{l+s+si}{\PYZcb{}}\PY{l+s+s2}{ saved as }\PY{l+s+si}{\PYZob{}}\PY{n}{model\PYZus{}name}\PY{l+s+si}{\PYZcb{}}\PY{l+s+s2}{\PYZus{}results.csv}\PY{l+s+s2}{\PYZdq{}}\PY{p}{)}

    \PY{c+c1}{\PYZsh{} Save the confusion matrix in a CSV file in the model folder}
    \PY{k}{if} \PY{n}{model\PYZus{}results}\PY{p}{[}\PY{l+s+s2}{\PYZdq{}}\PY{l+s+s2}{confusion\PYZus{}matrix}\PY{l+s+s2}{\PYZdq{}}\PY{p}{]} \PY{o+ow}{is} \PY{o+ow}{not} \PY{k+kc}{None}\PY{p}{:}
        \PY{n}{cm\PYZus{}df} \PY{o}{=} \PY{n}{pd}\PY{o}{.}\PY{n}{DataFrame}\PY{p}{(}\PY{n}{model\PYZus{}results}\PY{p}{[}\PY{l+s+s2}{\PYZdq{}}\PY{l+s+s2}{confusion\PYZus{}matrix}\PY{l+s+s2}{\PYZdq{}}\PY{p}{]}\PY{p}{,} \PY{n}{index}\PY{o}{=}\PY{p}{[}\PY{l+s+s1}{\PYZsq{}}\PY{l+s+s1}{True Negative}\PY{l+s+s1}{\PYZsq{}}\PY{p}{,} \PY{l+s+s1}{\PYZsq{}}\PY{l+s+s1}{True Positive}\PY{l+s+s1}{\PYZsq{}}\PY{p}{]}\PY{p}{,} \PY{n}{columns}\PY{o}{=}\PY{p}{[}\PY{l+s+s1}{\PYZsq{}}\PY{l+s+s1}{Predicted Negative}\PY{l+s+s1}{\PYZsq{}}\PY{p}{,} \PY{l+s+s1}{\PYZsq{}}\PY{l+s+s1}{Predicted Positive}\PY{l+s+s1}{\PYZsq{}}\PY{p}{]}\PY{p}{)}
        \PY{n}{cm\PYZus{}df}\PY{o}{.}\PY{n}{to\PYZus{}csv}\PY{p}{(}\PY{l+s+sa}{f}\PY{l+s+s1}{\PYZsq{}}\PY{l+s+si}{\PYZob{}}\PY{n}{model\PYZus{}dir}\PY{l+s+si}{\PYZcb{}}\PY{l+s+s1}{/}\PY{l+s+si}{\PYZob{}}\PY{n}{model\PYZus{}name}\PY{l+s+si}{\PYZcb{}}\PY{l+s+s1}{\PYZus{}confusion\PYZus{}matrix.csv}\PY{l+s+s1}{\PYZsq{}}\PY{p}{,} \PY{n}{index}\PY{o}{=}\PY{k+kc}{True}\PY{p}{)}
        \PY{n+nb}{print}\PY{p}{(}\PY{l+s+sa}{f}\PY{l+s+s2}{\PYZdq{}}\PY{l+s+se}{\PYZbs{}n}\PY{l+s+s2}{Confusion Matrix for }\PY{l+s+si}{\PYZob{}}\PY{n}{model\PYZus{}name}\PY{l+s+si}{\PYZcb{}}\PY{l+s+s2}{ saved as }\PY{l+s+si}{\PYZob{}}\PY{n}{model\PYZus{}name}\PY{l+s+si}{\PYZcb{}}\PY{l+s+s2}{\PYZus{}confusion\PYZus{}matrix.csv}\PY{l+s+s2}{\PYZdq{}}\PY{p}{)}

    \PY{c+c1}{\PYZsh{} save the model weights}
    \PY{n}{torch}\PY{o}{.}\PY{n}{save}\PY{p}{(}\PY{n}{model}\PY{o}{.}\PY{n}{state\PYZus{}dict}\PY{p}{(}\PY{p}{)}\PY{p}{,} \PY{l+s+sa}{f}\PY{l+s+s1}{\PYZsq{}}\PY{l+s+si}{\PYZob{}}\PY{n}{model\PYZus{}dir}\PY{l+s+si}{\PYZcb{}}\PY{l+s+s1}{/}\PY{l+s+si}{\PYZob{}}\PY{n}{model\PYZus{}name}\PY{l+s+si}{\PYZcb{}}\PY{l+s+s1}{\PYZus{}weights.pth}\PY{l+s+s1}{\PYZsq{}}\PY{p}{)}
    \PY{n+nb}{print}\PY{p}{(}\PY{l+s+sa}{f}\PY{l+s+s2}{\PYZdq{}}\PY{l+s+se}{\PYZbs{}n}\PY{l+s+s2}{Model weights for }\PY{l+s+si}{\PYZob{}}\PY{n}{model\PYZus{}name}\PY{l+s+si}{\PYZcb{}}\PY{l+s+s2}{ saved as }\PY{l+s+si}{\PYZob{}}\PY{n}{model\PYZus{}name}\PY{l+s+si}{\PYZcb{}}\PY{l+s+s2}{\PYZus{}weights.pth}\PY{l+s+s2}{\PYZdq{}}\PY{p}{)}

    \PY{k}{return} \PY{n}{model\PYZus{}results}
\end{Verbatim}
\end{tcolorbox}

    \subsection{Performance}\label{performance}

is a function that take the results of a model saved on drive and plot
the confusion matrix, the results of the last epoch and (train vs
validation) loss and accuracy, among all epoch

    \begin{tcolorbox}[breakable, size=fbox, boxrule=1pt, pad at break*=1mm,colback=cellbackground, colframe=cellborder]
\prompt{In}{incolor}{9}{\boxspacing}
\begin{Verbatim}[commandchars=\\\{\}]
\PY{k}{def} \PY{n+nf}{performance}\PY{p}{(}\PY{n}{nome\PYZus{}modello}\PY{p}{)}\PY{p}{:}
    \PY{c+c1}{\PYZsh{} Percorso per i file dei dati}
    \PY{n}{cm\PYZus{}file} \PY{o}{=} \PY{l+s+sa}{f}\PY{l+s+s1}{\PYZsq{}}\PY{l+s+s1}{/content/drive/MyDrive/deepfake\PYZus{}dataset/models/}\PY{l+s+si}{\PYZob{}}\PY{n}{nome\PYZus{}modello}\PY{l+s+si}{\PYZcb{}}\PY{l+s+s1}{/}\PY{l+s+si}{\PYZob{}}\PY{n}{nome\PYZus{}modello}\PY{l+s+si}{\PYZcb{}}\PY{l+s+s1}{\PYZus{}confusion\PYZus{}matrix.csv}\PY{l+s+s1}{\PYZsq{}}
    \PY{n}{res\PYZus{}file} \PY{o}{=} \PY{l+s+sa}{f}\PY{l+s+s1}{\PYZsq{}}\PY{l+s+s1}{/content/drive/MyDrive/deepfake\PYZus{}dataset/models/}\PY{l+s+si}{\PYZob{}}\PY{n}{nome\PYZus{}modello}\PY{l+s+si}{\PYZcb{}}\PY{l+s+s1}{/}\PY{l+s+si}{\PYZob{}}\PY{n}{nome\PYZus{}modello}\PY{l+s+si}{\PYZcb{}}\PY{l+s+s1}{\PYZus{}results.csv}\PY{l+s+s1}{\PYZsq{}}

    \PY{c+c1}{\PYZsh{} Carica i dati}
    \PY{n}{cm\PYZus{}resnet} \PY{o}{=} \PY{n}{pd}\PY{o}{.}\PY{n}{read\PYZus{}csv}\PY{p}{(}\PY{n}{cm\PYZus{}file}\PY{p}{)}
    \PY{n}{cm\PYZus{}resnet} \PY{o}{=} \PY{n}{cm\PYZus{}resnet}\PY{o}{.}\PY{n}{iloc}\PY{p}{[}\PY{p}{:}\PY{p}{,} \PY{l+m+mi}{1}\PY{p}{:}\PY{p}{]}  \PY{c+c1}{\PYZsh{} Rimuovi la prima colonna (indici)}

    \PY{n}{res\PYZus{}resnet} \PY{o}{=} \PY{n}{pd}\PY{o}{.}\PY{n}{read\PYZus{}csv}\PY{p}{(}\PY{n}{res\PYZus{}file}\PY{p}{)}
    \PY{c+c1}{\PYZsh{} Ottieni l\PYZsq{}ultima riga come dizionario}
    \PY{n}{last\PYZus{}row} \PY{o}{=} \PY{n}{res\PYZus{}resnet}\PY{o}{.}\PY{n}{iloc}\PY{p}{[}\PY{o}{\PYZhy{}}\PY{l+m+mi}{1}\PY{p}{]}\PY{o}{.}\PY{n}{to\PYZus{}dict}\PY{p}{(}\PY{p}{)}

    \PY{c+c1}{\PYZsh{} Converti i risultati in formato tabellare}
    \PY{n}{table\PYZus{}data} \PY{o}{=} \PY{p}{[}\PY{p}{(}\PY{n}{key}\PY{p}{,} \PY{l+s+sa}{f}\PY{l+s+s2}{\PYZdq{}}\PY{l+s+si}{\PYZob{}}\PY{n}{value}\PY{l+s+si}{:}\PY{l+s+s2}{.4f}\PY{l+s+si}{\PYZcb{}}\PY{l+s+s2}{\PYZdq{}} \PY{k}{if} \PY{n+nb}{isinstance}\PY{p}{(}\PY{n}{value}\PY{p}{,} \PY{n+nb}{float}\PY{p}{)} \PY{k}{else} \PY{n}{value}\PY{p}{)} \PY{k}{for} \PY{n}{key}\PY{p}{,} \PY{n}{value} \PY{o+ow}{in} \PY{n}{last\PYZus{}row}\PY{o}{.}\PY{n}{items}\PY{p}{(}\PY{p}{)}\PY{p}{]}
    \PY{n}{table\PYZus{}text} \PY{o}{=} \PY{n}{tabulate}\PY{p}{(}\PY{n}{table\PYZus{}data}\PY{p}{,} \PY{n}{headers}\PY{o}{=}\PY{p}{[}\PY{l+s+s2}{\PYZdq{}}\PY{l+s+s2}{Metric}\PY{l+s+s2}{\PYZdq{}}\PY{p}{,} \PY{l+s+s2}{\PYZdq{}}\PY{l+s+s2}{Value}\PY{l+s+s2}{\PYZdq{}}\PY{p}{]}\PY{p}{,} \PY{n}{tablefmt}\PY{o}{=}\PY{l+s+s2}{\PYZdq{}}\PY{l+s+s2}{grid}\PY{l+s+s2}{\PYZdq{}}\PY{p}{)}

    \PY{c+c1}{\PYZsh{} Prima figura (Confusion Matrix + Metrics)}
    \PY{n}{fig1}\PY{p}{,} \PY{n}{axes1} \PY{o}{=} \PY{n}{plt}\PY{o}{.}\PY{n}{subplots}\PY{p}{(}\PY{l+m+mi}{1}\PY{p}{,} \PY{l+m+mi}{2}\PY{p}{,} \PY{n}{figsize}\PY{o}{=}\PY{p}{(}\PY{l+m+mf}{9.3255}\PY{p}{,} \PY{l+m+mi}{5}\PY{p}{)}\PY{p}{,} \PY{n}{gridspec\PYZus{}kw}\PY{o}{=}\PY{p}{\PYZob{}}\PY{l+s+s1}{\PYZsq{}}\PY{l+s+s1}{width\PYZus{}ratios}\PY{l+s+s1}{\PYZsq{}}\PY{p}{:} \PY{p}{[}\PY{l+m+mi}{3}\PY{p}{,} \PY{l+m+mi}{3}\PY{p}{]}\PY{p}{\PYZcb{}}\PY{p}{)}

    \PY{c+c1}{\PYZsh{} Confusion Matrix}
    \PY{n}{sns}\PY{o}{.}\PY{n}{heatmap}\PY{p}{(}\PY{n}{cm\PYZus{}resnet}\PY{p}{,} \PY{n}{annot}\PY{o}{=}\PY{k+kc}{True}\PY{p}{,} \PY{n}{fmt}\PY{o}{=}\PY{l+s+s2}{\PYZdq{}}\PY{l+s+s2}{d}\PY{l+s+s2}{\PYZdq{}}\PY{p}{,} \PY{n}{cmap}\PY{o}{=}\PY{l+s+s2}{\PYZdq{}}\PY{l+s+s2}{YlGnBu}\PY{l+s+s2}{\PYZdq{}}\PY{p}{,} \PY{n}{cbar}\PY{o}{=}\PY{k+kc}{False}\PY{p}{,}
                \PY{n}{xticklabels}\PY{o}{=}\PY{p}{[}\PY{l+s+s1}{\PYZsq{}}\PY{l+s+s1}{Predicted Real}\PY{l+s+s1}{\PYZsq{}}\PY{p}{,} \PY{l+s+s1}{\PYZsq{}}\PY{l+s+s1}{Predicted Fake}\PY{l+s+s1}{\PYZsq{}}\PY{p}{]}\PY{p}{,}
                \PY{n}{yticklabels}\PY{o}{=}\PY{p}{[}\PY{l+s+s1}{\PYZsq{}}\PY{l+s+s1}{True Real}\PY{l+s+s1}{\PYZsq{}}\PY{p}{,} \PY{l+s+s1}{\PYZsq{}}\PY{l+s+s1}{True Fake}\PY{l+s+s1}{\PYZsq{}}\PY{p}{]}\PY{p}{,} \PY{n}{ax}\PY{o}{=}\PY{n}{axes1}\PY{p}{[}\PY{l+m+mi}{0}\PY{p}{]}\PY{p}{,}
                \PY{n}{linewidths}\PY{o}{=}\PY{l+m+mf}{0.5}\PY{p}{,} \PY{n}{linecolor}\PY{o}{=}\PY{l+s+s1}{\PYZsq{}}\PY{l+s+s1}{black}\PY{l+s+s1}{\PYZsq{}}\PY{p}{,} \PY{n}{square}\PY{o}{=}\PY{k+kc}{True}\PY{p}{)}
    \PY{n}{axes1}\PY{p}{[}\PY{l+m+mi}{0}\PY{p}{]}\PY{o}{.}\PY{n}{set\PYZus{}title}\PY{p}{(}\PY{l+s+s2}{\PYZdq{}}\PY{l+s+s2}{Confusion Matrix}\PY{l+s+s2}{\PYZdq{}}\PY{p}{)}

    \PY{c+c1}{\PYZsh{} Tabella con le metriche}
    \PY{n}{axes1}\PY{p}{[}\PY{l+m+mi}{1}\PY{p}{]}\PY{o}{.}\PY{n}{axis}\PY{p}{(}\PY{l+s+s1}{\PYZsq{}}\PY{l+s+s1}{off}\PY{l+s+s1}{\PYZsq{}}\PY{p}{)}  \PY{c+c1}{\PYZsh{} Disabilita gli assi}
    \PY{n}{axes1}\PY{p}{[}\PY{l+m+mi}{1}\PY{p}{]}\PY{o}{.}\PY{n}{text}\PY{p}{(}\PY{l+m+mi}{0}\PY{p}{,} \PY{l+m+mf}{0.5}\PY{p}{,} \PY{n}{table\PYZus{}text}\PY{p}{,} \PY{n}{fontsize}\PY{o}{=}\PY{l+m+mi}{10}\PY{p}{,} \PY{n}{fontfamily}\PY{o}{=}\PY{l+s+s1}{\PYZsq{}}\PY{l+s+s1}{monospace}\PY{l+s+s1}{\PYZsq{}}\PY{p}{,} \PY{n}{ha}\PY{o}{=}\PY{l+s+s1}{\PYZsq{}}\PY{l+s+s1}{left}\PY{l+s+s1}{\PYZsq{}}\PY{p}{,} \PY{n}{va}\PY{o}{=}\PY{l+s+s1}{\PYZsq{}}\PY{l+s+s1}{center}\PY{l+s+s1}{\PYZsq{}}\PY{p}{)}
    \PY{n}{axes1}\PY{p}{[}\PY{l+m+mi}{1}\PY{p}{]}\PY{o}{.}\PY{n}{set\PYZus{}title}\PY{p}{(}\PY{l+s+s2}{\PYZdq{}}\PY{l+s+s2}{From Last Epoch:}\PY{l+s+s2}{\PYZdq{}}\PY{p}{)}

    \PY{n}{plt}\PY{o}{.}\PY{n}{tight\PYZus{}layout}\PY{p}{(}\PY{p}{)}
    \PY{n}{plt}\PY{o}{.}\PY{n}{subplots\PYZus{}adjust}\PY{p}{(}\PY{n}{wspace}\PY{o}{=}\PY{l+m+mf}{0.6}\PY{p}{,} \PY{n}{left}\PY{o}{=}\PY{l+m+mf}{0.1}\PY{p}{,} \PY{n}{right}\PY{o}{=}\PY{l+m+mf}{0.9}\PY{p}{,} \PY{n}{top}\PY{o}{=}\PY{l+m+mf}{0.8}\PY{p}{,} \PY{n}{bottom}\PY{o}{=}\PY{l+m+mf}{0.2}\PY{p}{)}  \PY{c+c1}{\PYZsh{} Più spazio tra i subplot e margini}
    \PY{n}{plt}\PY{o}{.}\PY{n}{show}\PY{p}{(}\PY{p}{)}

    \PY{c+c1}{\PYZsh{} Seconda figura (Train Loss vs Validation Loss)}
    \PY{n}{fig2}\PY{p}{,} \PY{n}{axes2} \PY{o}{=} \PY{n}{plt}\PY{o}{.}\PY{n}{subplots}\PY{p}{(}\PY{l+m+mi}{1}\PY{p}{,} \PY{l+m+mi}{1}\PY{p}{,} \PY{n}{figsize}\PY{o}{=}\PY{p}{(}\PY{l+m+mi}{8}\PY{p}{,} \PY{l+m+mi}{5}\PY{p}{)}\PY{p}{)}  \PY{c+c1}{\PYZsh{} Un solo plot per Train/Validation Loss}
    \PY{n}{axes2}\PY{o}{.}\PY{n}{plot}\PY{p}{(}\PY{n}{res\PYZus{}resnet}\PY{p}{[}\PY{l+s+s1}{\PYZsq{}}\PY{l+s+s1}{Epoch}\PY{l+s+s1}{\PYZsq{}}\PY{p}{]}\PY{p}{,} \PY{n}{res\PYZus{}resnet}\PY{p}{[}\PY{l+s+s1}{\PYZsq{}}\PY{l+s+s1}{Train Loss}\PY{l+s+s1}{\PYZsq{}}\PY{p}{]}\PY{p}{,} \PY{n}{label}\PY{o}{=}\PY{l+s+s1}{\PYZsq{}}\PY{l+s+s1}{Train Loss}\PY{l+s+s1}{\PYZsq{}}\PY{p}{,} \PY{n}{color}\PY{o}{=}\PY{l+s+s1}{\PYZsq{}}\PY{l+s+s1}{blue}\PY{l+s+s1}{\PYZsq{}}\PY{p}{)}
    \PY{n}{axes2}\PY{o}{.}\PY{n}{plot}\PY{p}{(}\PY{n}{res\PYZus{}resnet}\PY{p}{[}\PY{l+s+s1}{\PYZsq{}}\PY{l+s+s1}{Epoch}\PY{l+s+s1}{\PYZsq{}}\PY{p}{]}\PY{p}{,} \PY{n}{res\PYZus{}resnet}\PY{p}{[}\PY{l+s+s1}{\PYZsq{}}\PY{l+s+s1}{Validation Loss}\PY{l+s+s1}{\PYZsq{}}\PY{p}{]}\PY{p}{,} \PY{n}{label}\PY{o}{=}\PY{l+s+s1}{\PYZsq{}}\PY{l+s+s1}{Validation Loss}\PY{l+s+s1}{\PYZsq{}}\PY{p}{,} \PY{n}{color}\PY{o}{=}\PY{l+s+s1}{\PYZsq{}}\PY{l+s+s1}{orange}\PY{l+s+s1}{\PYZsq{}}\PY{p}{)}
    \PY{n}{axes2}\PY{o}{.}\PY{n}{set\PYZus{}title}\PY{p}{(}\PY{l+s+s2}{\PYZdq{}}\PY{l+s+s2}{Train vs Validation Loss}\PY{l+s+s2}{\PYZdq{}}\PY{p}{)}
    \PY{n}{axes2}\PY{o}{.}\PY{n}{set\PYZus{}xlabel}\PY{p}{(}\PY{l+s+s2}{\PYZdq{}}\PY{l+s+s2}{Epoch}\PY{l+s+s2}{\PYZdq{}}\PY{p}{)}
    \PY{n}{axes2}\PY{o}{.}\PY{n}{set\PYZus{}ylabel}\PY{p}{(}\PY{l+s+s2}{\PYZdq{}}\PY{l+s+s2}{Loss}\PY{l+s+s2}{\PYZdq{}}\PY{p}{)}
    \PY{n}{axes2}\PY{o}{.}\PY{n}{legend}\PY{p}{(}\PY{p}{)}
    \PY{n}{axes2}\PY{o}{.}\PY{n}{grid}\PY{p}{(}\PY{k+kc}{True}\PY{p}{,} \PY{n}{linestyle}\PY{o}{=}\PY{l+s+s1}{\PYZsq{}}\PY{l+s+s1}{\PYZhy{}\PYZhy{}}\PY{l+s+s1}{\PYZsq{}}\PY{p}{,} \PY{n}{alpha}\PY{o}{=}\PY{l+m+mf}{0.6}\PY{p}{)}
    \PY{n}{axes2}\PY{o}{.}\PY{n}{set\PYZus{}xticks}\PY{p}{(}\PY{n+nb}{range}\PY{p}{(}\PY{l+m+mi}{1}\PY{p}{,} \PY{n+nb}{len}\PY{p}{(}\PY{n}{res\PYZus{}resnet}\PY{p}{)}\PY{p}{,} \PY{l+m+mi}{2}\PY{p}{)}\PY{p}{)}
    \PY{n}{plt}\PY{o}{.}\PY{n}{tight\PYZus{}layout}\PY{p}{(}\PY{p}{)}
    \PY{n}{plt}\PY{o}{.}\PY{n}{show}\PY{p}{(}\PY{p}{)}

    \PY{c+c1}{\PYZsh{} Terza figura (Train Accuracy vs Validation Accuracy)}
    \PY{n}{fig3}\PY{p}{,} \PY{n}{axes3} \PY{o}{=} \PY{n}{plt}\PY{o}{.}\PY{n}{subplots}\PY{p}{(}\PY{l+m+mi}{1}\PY{p}{,} \PY{l+m+mi}{1}\PY{p}{,} \PY{n}{figsize}\PY{o}{=}\PY{p}{(}\PY{l+m+mi}{8}\PY{p}{,} \PY{l+m+mi}{5}\PY{p}{)}\PY{p}{)}  \PY{c+c1}{\PYZsh{} Un solo plot per Train/Validation Accuracy}
    \PY{n}{axes3}\PY{o}{.}\PY{n}{plot}\PY{p}{(}\PY{n}{res\PYZus{}resnet}\PY{p}{[}\PY{l+s+s1}{\PYZsq{}}\PY{l+s+s1}{Epoch}\PY{l+s+s1}{\PYZsq{}}\PY{p}{]}\PY{p}{,} \PY{n}{res\PYZus{}resnet}\PY{p}{[}\PY{l+s+s1}{\PYZsq{}}\PY{l+s+s1}{Train Accuracy}\PY{l+s+s1}{\PYZsq{}}\PY{p}{]}\PY{p}{,} \PY{n}{label}\PY{o}{=}\PY{l+s+s1}{\PYZsq{}}\PY{l+s+s1}{Train Accuracy}\PY{l+s+s1}{\PYZsq{}}\PY{p}{,} \PY{n}{color}\PY{o}{=}\PY{l+s+s1}{\PYZsq{}}\PY{l+s+s1}{green}\PY{l+s+s1}{\PYZsq{}}\PY{p}{)}
    \PY{n}{axes3}\PY{o}{.}\PY{n}{plot}\PY{p}{(}\PY{n}{res\PYZus{}resnet}\PY{p}{[}\PY{l+s+s1}{\PYZsq{}}\PY{l+s+s1}{Epoch}\PY{l+s+s1}{\PYZsq{}}\PY{p}{]}\PY{p}{,} \PY{n}{res\PYZus{}resnet}\PY{p}{[}\PY{l+s+s1}{\PYZsq{}}\PY{l+s+s1}{Validation Accuracy}\PY{l+s+s1}{\PYZsq{}}\PY{p}{]}\PY{p}{,} \PY{n}{label}\PY{o}{=}\PY{l+s+s1}{\PYZsq{}}\PY{l+s+s1}{Validation Accuracy}\PY{l+s+s1}{\PYZsq{}}\PY{p}{,} \PY{n}{color}\PY{o}{=}\PY{l+s+s1}{\PYZsq{}}\PY{l+s+s1}{red}\PY{l+s+s1}{\PYZsq{}}\PY{p}{)}
    \PY{n}{axes3}\PY{o}{.}\PY{n}{set\PYZus{}title}\PY{p}{(}\PY{l+s+s2}{\PYZdq{}}\PY{l+s+s2}{Train vs Validation Accuracy}\PY{l+s+s2}{\PYZdq{}}\PY{p}{)}
    \PY{n}{axes3}\PY{o}{.}\PY{n}{set\PYZus{}xlabel}\PY{p}{(}\PY{l+s+s2}{\PYZdq{}}\PY{l+s+s2}{Epoch}\PY{l+s+s2}{\PYZdq{}}\PY{p}{)}
    \PY{n}{axes3}\PY{o}{.}\PY{n}{set\PYZus{}ylabel}\PY{p}{(}\PY{l+s+s2}{\PYZdq{}}\PY{l+s+s2}{Accuracy}\PY{l+s+s2}{\PYZdq{}}\PY{p}{)}
    \PY{n}{axes3}\PY{o}{.}\PY{n}{legend}\PY{p}{(}\PY{p}{)}
    \PY{n}{axes3}\PY{o}{.}\PY{n}{grid}\PY{p}{(}\PY{k+kc}{True}\PY{p}{,} \PY{n}{linestyle}\PY{o}{=}\PY{l+s+s1}{\PYZsq{}}\PY{l+s+s1}{\PYZhy{}\PYZhy{}}\PY{l+s+s1}{\PYZsq{}}\PY{p}{,} \PY{n}{alpha}\PY{o}{=}\PY{l+m+mf}{0.6}\PY{p}{)}
    \PY{n}{axes3}\PY{o}{.}\PY{n}{set\PYZus{}xticks}\PY{p}{(}\PY{n+nb}{range}\PY{p}{(}\PY{l+m+mi}{1}\PY{p}{,} \PY{n+nb}{len}\PY{p}{(}\PY{n}{res\PYZus{}resnet}\PY{p}{)}\PY{p}{,} \PY{l+m+mi}{2}\PY{p}{)}\PY{p}{)}
    \PY{n}{plt}\PY{o}{.}\PY{n}{tight\PYZus{}layout}\PY{p}{(}\PY{p}{)}
    \PY{n}{plt}\PY{o}{.}\PY{n}{show}\PY{p}{(}\PY{p}{)}
\end{Verbatim}
\end{tcolorbox}

    \subsection{Recap the Results}\label{recap-the-results}

    \begin{tcolorbox}[breakable, size=fbox, boxrule=1pt, pad at break*=1mm,colback=cellbackground, colframe=cellborder]
\prompt{In}{incolor}{1}{\boxspacing}
\begin{Verbatim}[commandchars=\\\{\}]
\PY{k}{def} \PY{n+nf}{recap}\PY{p}{(}\PY{n}{model\PYZus{}names}\PY{p}{)}\PY{p}{:}
    
    \PY{n}{models\PYZus{}dir} \PY{o}{=} \PY{l+s+s1}{\PYZsq{}}\PY{l+s+s1}{/content/drive/MyDrive/deepfake\PYZus{}dataset/models}\PY{l+s+s1}{\PYZsq{}}
    \PY{n}{data} \PY{o}{=} \PY{p}{[}\PY{p}{]}
    
    \PY{k}{for} \PY{n}{model} \PY{o+ow}{in} \PY{n}{model\PYZus{}names}\PY{p}{:}
        
        \PY{n}{results\PYZus{}file} \PY{o}{=} \PY{n}{os}\PY{o}{.}\PY{n}{path}\PY{o}{.}\PY{n}{join}\PY{p}{(}\PY{n}{models\PYZus{}dir}\PY{p}{,} \PY{n}{model}\PY{p}{,} \PY{l+s+sa}{f}\PY{l+s+s1}{\PYZsq{}}\PY{l+s+si}{\PYZob{}}\PY{n}{model}\PY{l+s+si}{\PYZcb{}}\PY{l+s+s1}{\PYZus{}results.csv}\PY{l+s+s1}{\PYZsq{}}\PY{p}{)}
        \PY{k}{if} \PY{n}{os}\PY{o}{.}\PY{n}{path}\PY{o}{.}\PY{n}{exists}\PY{p}{(}\PY{n}{results\PYZus{}file}\PY{p}{)}\PY{p}{:}
            \PY{n}{df} \PY{o}{=} \PY{n}{pd}\PY{o}{.}\PY{n}{read\PYZus{}csv}\PY{p}{(}\PY{n}{results\PYZus{}file}\PY{p}{)}

            \PY{k}{if} \PY{o+ow}{not} \PY{n}{df}\PY{o}{.}\PY{n}{empty}\PY{p}{:}
                \PY{n}{last\PYZus{}row} \PY{o}{=} \PY{n}{df}\PY{o}{.}\PY{n}{iloc}\PY{p}{[}\PY{o}{\PYZhy{}}\PY{l+m+mi}{1}\PY{p}{]}

                \PY{n}{t\PYZus{}loss} \PY{o}{=} \PY{n}{last\PYZus{}row}\PY{o}{.}\PY{n}{get}\PY{p}{(}\PY{l+s+s1}{\PYZsq{}}\PY{l+s+s1}{Train Loss}\PY{l+s+s1}{\PYZsq{}}\PY{p}{,} \PY{k+kc}{None}\PY{p}{)}
                \PY{n}{t\PYZus{}acc} \PY{o}{=} \PY{n}{last\PYZus{}row}\PY{o}{.}\PY{n}{get}\PY{p}{(}\PY{l+s+s1}{\PYZsq{}}\PY{l+s+s1}{Train Accuracy}\PY{l+s+s1}{\PYZsq{}}\PY{p}{,} \PY{k+kc}{None}\PY{p}{)}
                \PY{n}{v\PYZus{}loss} \PY{o}{=} \PY{n}{last\PYZus{}row}\PY{o}{.}\PY{n}{get}\PY{p}{(}\PY{l+s+s1}{\PYZsq{}}\PY{l+s+s1}{Validation Loss}\PY{l+s+s1}{\PYZsq{}}\PY{p}{,} \PY{k+kc}{None}\PY{p}{)}
                \PY{n}{v\PYZus{}acc} \PY{o}{=} \PY{n}{last\PYZus{}row}\PY{o}{.}\PY{n}{get}\PY{p}{(}\PY{l+s+s1}{\PYZsq{}}\PY{l+s+s1}{Validation Accuracy}\PY{l+s+s1}{\PYZsq{}}\PY{p}{,} \PY{k+kc}{None}\PY{p}{)}
                \PY{n}{precision} \PY{o}{=} \PY{n}{last\PYZus{}row}\PY{o}{.}\PY{n}{get}\PY{p}{(}\PY{l+s+s1}{\PYZsq{}}\PY{l+s+s1}{Precision}\PY{l+s+s1}{\PYZsq{}}\PY{p}{,} \PY{k+kc}{None}\PY{p}{)}
                \PY{n}{recall} \PY{o}{=} \PY{n}{last\PYZus{}row}\PY{o}{.}\PY{n}{get}\PY{p}{(}\PY{l+s+s1}{\PYZsq{}}\PY{l+s+s1}{Recall}\PY{l+s+s1}{\PYZsq{}}\PY{p}{,} \PY{k+kc}{None}\PY{p}{)}
                \PY{n}{f1} \PY{o}{=} \PY{n}{last\PYZus{}row}\PY{o}{.}\PY{n}{get}\PY{p}{(}\PY{l+s+s1}{\PYZsq{}}\PY{l+s+s1}{F1 Score}\PY{l+s+s1}{\PYZsq{}}\PY{p}{,} \PY{k+kc}{None}\PY{p}{)}

                \PY{n}{data}\PY{o}{.}\PY{n}{append}\PY{p}{(}\PY{p}{[}\PY{n}{model}\PY{p}{,} \PY{n}{t\PYZus{}loss}\PY{p}{,} \PY{n}{t\PYZus{}acc}\PY{p}{,} \PY{n}{v\PYZus{}loss}\PY{p}{,} \PY{n}{v\PYZus{}acc}\PY{p}{,} \PY{n}{precision}\PY{p}{,} \PY{n}{recall}\PY{p}{,} \PY{n}{f1}\PY{p}{]}\PY{p}{)}
            \PY{k}{else}\PY{p}{:}
                \PY{n+nb}{print}\PY{p}{(}\PY{l+s+sa}{f}\PY{l+s+s2}{\PYZdq{}}\PY{l+s+s2}{Il file }\PY{l+s+si}{\PYZob{}}\PY{n}{results\PYZus{}file}\PY{l+s+si}{\PYZcb{}}\PY{l+s+s2}{ è vuoto.}\PY{l+s+s2}{\PYZdq{}}\PY{p}{)}
        \PY{k}{else}\PY{p}{:}
            \PY{n+nb}{print}\PY{p}{(}\PY{l+s+sa}{f}\PY{l+s+s2}{\PYZdq{}}\PY{l+s+s2}{Il file }\PY{l+s+si}{\PYZob{}}\PY{n}{results\PYZus{}file}\PY{l+s+si}{\PYZcb{}}\PY{l+s+s2}{ non esiste.}\PY{l+s+s2}{\PYZdq{}}\PY{p}{)}

    \PY{n}{columns} \PY{o}{=} \PY{p}{[}\PY{l+s+s1}{\PYZsq{}}\PY{l+s+s1}{Model}\PY{l+s+s1}{\PYZsq{}}\PY{p}{,} \PY{l+s+s1}{\PYZsq{}}\PY{l+s+s1}{Train Loss}\PY{l+s+s1}{\PYZsq{}}\PY{p}{,} \PY{l+s+s1}{\PYZsq{}}\PY{l+s+s1}{Train Accuracy}\PY{l+s+s1}{\PYZsq{}}\PY{p}{,} \PY{l+s+s1}{\PYZsq{}}\PY{l+s+s1}{Validation Loss}\PY{l+s+s1}{\PYZsq{}}\PY{p}{,}\PY{l+s+s1}{\PYZsq{}}\PY{l+s+s1}{Validation Accuracy}\PY{l+s+s1}{\PYZsq{}}\PY{p}{,} \PY{l+s+s1}{\PYZsq{}}\PY{l+s+s1}{Precision}\PY{l+s+s1}{\PYZsq{}}\PY{p}{,} \PY{l+s+s1}{\PYZsq{}}\PY{l+s+s1}{Recall}\PY{l+s+s1}{\PYZsq{}}\PY{p}{,} \PY{l+s+s1}{\PYZsq{}}\PY{l+s+s1}{F1 Score}\PY{l+s+s1}{\PYZsq{}}\PY{p}{]}
    \PY{n}{df\PYZus{}summary} \PY{o}{=} \PY{n}{pd}\PY{o}{.}\PY{n}{DataFrame}\PY{p}{(}\PY{n}{data}\PY{p}{,} \PY{n}{columns}\PY{o}{=}\PY{n}{columns}\PY{p}{)}

    \PY{n}{fig}\PY{p}{,} \PY{n}{ax} \PY{o}{=} \PY{n}{plt}\PY{o}{.}\PY{n}{subplots}\PY{p}{(}\PY{n}{figsize}\PY{o}{=}\PY{p}{(}\PY{l+m+mi}{8}\PY{p}{,} \PY{l+m+mf}{1.7}\PY{p}{)}\PY{p}{)}
    \PY{n}{ax}\PY{o}{.}\PY{n}{axis}\PY{p}{(}\PY{l+s+s1}{\PYZsq{}}\PY{l+s+s1}{off}\PY{l+s+s1}{\PYZsq{}}\PY{p}{)}
    \PY{n}{ax}\PY{o}{.}\PY{n}{axis}\PY{p}{(}\PY{l+s+s1}{\PYZsq{}}\PY{l+s+s1}{tight}\PY{l+s+s1}{\PYZsq{}}\PY{p}{)}

    \PY{n}{table} \PY{o}{=} \PY{n}{ax}\PY{o}{.}\PY{n}{table}\PY{p}{(}\PY{n}{cellText}\PY{o}{=}\PY{n}{df\PYZus{}summary}\PY{o}{.}\PY{n}{values}\PY{p}{,} \PY{n}{colLabels}\PY{o}{=}\PY{n}{df\PYZus{}summary}\PY{o}{.}\PY{n}{columns}\PY{p}{,} \PY{n}{loc}\PY{o}{=}\PY{l+s+s1}{\PYZsq{}}\PY{l+s+s1}{center}\PY{l+s+s1}{\PYZsq{}}\PY{p}{,} \PY{n}{cellLoc}\PY{o}{=}\PY{l+s+s1}{\PYZsq{}}\PY{l+s+s1}{center}\PY{l+s+s1}{\PYZsq{}}\PY{p}{)}
    \PY{n}{table}\PY{o}{.}\PY{n}{auto\PYZus{}set\PYZus{}font\PYZus{}size}\PY{p}{(}\PY{k+kc}{False}\PY{p}{)}
    \PY{n}{table}\PY{o}{.}\PY{n}{set\PYZus{}fontsize}\PY{p}{(}\PY{l+m+mi}{10}\PY{p}{)}
    \PY{n}{table}\PY{o}{.}\PY{n}{auto\PYZus{}set\PYZus{}column\PYZus{}width}\PY{p}{(}\PY{n}{col}\PY{o}{=}\PY{n+nb}{list}\PY{p}{(}\PY{n+nb}{range}\PY{p}{(}\PY{n+nb}{len}\PY{p}{(}\PY{n}{columns}\PY{p}{)}\PY{p}{)}\PY{p}{)}\PY{p}{)}  \PY{c+c1}{\PYZsh{} Adatta larghezza colonne automaticamente}

    \PY{n}{plt}\PY{o}{.}\PY{n}{tight\PYZus{}layout}\PY{p}{(}\PY{p}{)}
    \PY{n}{plt}\PY{o}{.}\PY{n}{show}\PY{p}{(}\PY{p}{)}
\end{Verbatim}
\end{tcolorbox}

    \section{Libraries}\label{libraries}

    \begin{tcolorbox}[breakable, size=fbox, boxrule=1pt, pad at break*=1mm,colback=cellbackground, colframe=cellborder]
\prompt{In}{incolor}{35}{\boxspacing}
\begin{Verbatim}[commandchars=\\\{\}]
\PY{k+kn}{from} \PY{n+nn}{PIL} \PY{k+kn}{import} \PY{n}{Image}
\PY{k+kn}{import} \PY{n+nn}{os}
\PY{k+kn}{import} \PY{n+nn}{cv2}
\PY{k+kn}{import} \PY{n+nn}{zipfile}
\PY{k+kn}{import} \PY{n+nn}{numpy} \PY{k}{as} \PY{n+nn}{np}
\PY{k+kn}{from} \PY{n+nn}{collections} \PY{k+kn}{import} \PY{n}{Counter}
\PY{k+kn}{from} \PY{n+nn}{IPython}\PY{n+nn}{.}\PY{n+nn}{display} \PY{k+kn}{import} \PY{n}{HTML}\PY{p}{,} \PY{n}{display}

\PY{k+kn}{import} \PY{n+nn}{torch}
\PY{k+kn}{from} \PY{n+nn}{torch}\PY{n+nn}{.}\PY{n+nn}{utils}\PY{n+nn}{.}\PY{n+nn}{data} \PY{k+kn}{import} \PY{n}{DataLoader}\PY{p}{,} \PY{n}{random\PYZus{}split}
\PY{k+kn}{from} \PY{n+nn}{torchvision} \PY{k+kn}{import} \PY{n}{datasets}\PY{p}{,} \PY{n}{transforms}\PY{p}{,} \PY{n}{models}
\PY{k+kn}{from} \PY{n+nn}{torchvision}\PY{n+nn}{.}\PY{n+nn}{models} \PY{k+kn}{import} \PY{n}{vit\PYZus{}b\PYZus{}16}\PY{p}{,} \PY{n}{resnet18}\PY{p}{,} \PY{n}{ResNet18\PYZus{}Weights}\PY{p}{,}\PY{n}{efficientnet\PYZus{}b0}\PY{p}{,} \PY{n}{EfficientNet\PYZus{}B0\PYZus{}Weights}\PY{p}{,} \PY{n}{densenet121}\PY{p}{,} \PY{n}{DenseNet121\PYZus{}Weights}
\PY{k+kn}{import} \PY{n+nn}{torch}\PY{n+nn}{.}\PY{n+nn}{nn} \PY{k}{as} \PY{n+nn}{nn}
\PY{k+kn}{import} \PY{n+nn}{torch}\PY{n+nn}{.}\PY{n+nn}{optim} \PY{k}{as} \PY{n+nn}{optim}
\PY{k+kn}{from} \PY{n+nn}{tqdm}\PY{n+nn}{.}\PY{n+nn}{notebook} \PY{k+kn}{import} \PY{n}{tqdm}

\PY{k+kn}{from} \PY{n+nn}{transformers} \PY{k+kn}{import} \PY{n}{ViTForImageClassification}

\PY{k+kn}{import} \PY{n+nn}{torch}\PY{n+nn}{.}\PY{n+nn}{nn} \PY{k}{as} \PY{n+nn}{nn}
\PY{k+kn}{import} \PY{n+nn}{torch}\PY{n+nn}{.}\PY{n+nn}{optim} \PY{k}{as} \PY{n+nn}{optim}
\PY{k+kn}{from} \PY{n+nn}{tqdm}\PY{n+nn}{.}\PY{n+nn}{notebook} \PY{k+kn}{import} \PY{n}{tqdm}

\PY{k+kn}{import} \PY{n+nn}{pandas} \PY{k}{as} \PY{n+nn}{pd}
\PY{k+kn}{import} \PY{n+nn}{matplotlib}\PY{n+nn}{.}\PY{n+nn}{pyplot} \PY{k}{as} \PY{n+nn}{plt}
\PY{k+kn}{import} \PY{n+nn}{seaborn} \PY{k}{as} \PY{n+nn}{sns}
\PY{k+kn}{import} \PY{n+nn}{plotly}\PY{n+nn}{.}\PY{n+nn}{graph\PYZus{}objects} \PY{k}{as} \PY{n+nn}{go}
\PY{k+kn}{from} \PY{n+nn}{sklearn}\PY{n+nn}{.}\PY{n+nn}{metrics} \PY{k+kn}{import} \PY{n}{confusion\PYZus{}matrix}\PY{p}{,} \PY{n}{precision\PYZus{}score}\PY{p}{,} \PY{n}{recall\PYZus{}score}\PY{p}{,}\PY{n}{f1\PYZus{}score}\PY{p}{,} \PY{n}{accuracy\PYZus{}score}
\PY{k+kn}{import} \PY{n+nn}{glob}
\PY{k+kn}{from} \PY{n+nn}{tabulate} \PY{k+kn}{import} \PY{n}{tabulate}
\PY{k+kn}{from} \PY{n+nn}{IPython}\PY{n+nn}{.}\PY{n+nn}{display} \PY{k+kn}{import} \PY{n}{Image} \PY{k}{as} \PY{n}{ipydis}
\PY{k+kn}{from} \PY{n+nn}{PIL} \PY{k+kn}{import} \PY{n}{Image}
\end{Verbatim}
\end{tcolorbox}


    % Add a bibliography block to the postdoc
    
    
    
\end{document}
